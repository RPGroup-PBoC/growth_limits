\subsection{Rapid Growth Requires \textit{E. coli} to Increase Both Cell Size and Ribosomal
Mass Fraction}
In the right-hand side of \FIG{ribosome_limit}(B),  we also find that above about 0.75
hr$^{-1}$, the growth rate is determined solely by the ribosomal mass fraction
$\Phi_R$, since $f_a$ is close to 1, and $r_t$ is near its maximal rate
\citep{dai2016}. While $\Phi_R$ will need to increase in order for cells to
grow faster, the fractional dependence in \EQ{lam_limited}
gives little insight into how this scaling is actually achieved by the cell.

It is now well-documented that \textit{E. coli} cells add a constant volume per
origin of replication, which is robust to a remarkable array of cellular
perturbations \citep{si2017}. Given the proteomic measurements featured in this
work, we find that the ribosome copy number also scales in proportion to
$\langle$\# ori$\rangle$ (\FIG{translation_ecoli_partA}(A)). However,  an
increase in ribosome abundance alone is not necessarily sufficient to increase
growth rate and  we also need to consider how $\Phi_R$ varies with $\langle$\#
ori$\rangle$. Importantly, as shown in \FIG{translation_ecoli_partA}(B), we find
that the deviations in protein expression with $\langle$\# ori$\rangle$ are
largely restricted to regions of ribosomal protein genes
\FIG{translation_ecoli_partA}(B). Here we have calculated the position-dependent
protein expression across the chromosome by a running Gaussian average of
protein copy number (20 kbp st. dev. averaging window) based on each gene's
transcriptional start site. These were median-subtracted to account for the
change in total protein abundance with $\langle$\# ori$\rangle$. This result
suggests that $\Phi_R$ is also being tuned in proportion to $\langle$\#
ori$\rangle$ under nutrient-limited growth, and in particular, it is through
this additional dependence on $\Phi_R$, combined with the exponential increase
in $\langle$\# ori$\rangle$, that \textit{E. coli} exhibits an exponential
increase in cell size with growth rate.


\begin{figure*}
    \begin{fullwidth}
    \centering{
        \includegraphics{main_figs/fig10_ribosome_growth_limit_ecoli_a_polar_coord.pdf}
        \caption{\textbf{Cells increase both absolute ribosome abundance and $\Phi_R$ with
        $\langle$\# ori$\rangle$.} (A) Plot of the ribosome copy number estimated from the
        proteomic data against the estimated $\langle$\# ori$\rangle$ (see Appendix
        Section "Estimation of $\langle$\# ori$\rangle$/ $\langle$\# ter$\rangle$ and $\langle$\# ori$\rangle$ for additional details). (B) A running
        Gaussian average (20 kbp st. dev.) of protein copy number is calculated
        for each growth condition considered by \citep{schmidt2016} based
        on each gene's transcriptional start site. Since total
        protein abundance increases with growth rate, protein copy numbers are
        median-subtracted to allow comparison between growth conditions.
        $\langle$\# ori$\rangle$ are estimated using the data in (A) and
        Equation \ref{eq:Nori}. } \label{fig:translation_ecoli_partA}
    }
    \end{fullwidth}
\end{figure*}
