\section{Derivation of Minimal Model for Nutrient-Mediated Growth Rate Control}
\label{sec:SI_model}
Here we provide a derivation of the minimal model for growth rate control under
nutrient-limited growth. By growth rate control, we are specifically referring
to the ability of bacteria to modulate their proteome ($N_\text{pep}$, $R$,
$\Phi_R$) and cell size as nutrient conditions change, with slower growing cells
generally being smaller in size \citep{ojkic2019}. This capability provides
bacteria with  a particular benefit when nutrients are more scarce since it will
mean there is a  smaller net demand on carbon, phosphorus, sulfur, and nitrogen.
The specific goal  of developing this model is to help us better explore the
overall constraints on  growth that follow from 1) our observation that many  of
the cellular processes we've considered require increased protein abundance at
faster growth rates, and 2) a strict limit on growth rate that is
governed by the ribosomal synthesis rate and ribosomal mass fraction $\Phi_R$.

In \textbf{\textit{Figure 12}}(A) of the main text we provide a schematic of the
model, where we consider growth as simply governed by the rate of protein
synthesis ($r_t \times R \times f_a$). In order to grow rapidly, at least to the
extent possible, these three parameters need to be maximized (with $r_t \leq$ 17
amino acids per second, and $f_a \leq$ 1 reported in the work of
\cite{dai2016}). The elongation rate $r_t$ will depend on how quickly
ribosomes can match codons with their correct amino-acyl tRNA, along with the
subsequent steps of peptide bond formation and translocation. This ultimately
depends on the cellular concentration amino acids, which we treat as a single
effective species, $[AA]_\text{eff}$.

In our model, we need to determine the rate of peptide elongation $r_t$, which we
consider as simply depending on the supply of amino acids (and,
therefore, also amino-acyl tRNAs) through a parameter $r_{AA}$ in units of AA
per second, and the rate of amino acid consumption by protein synthesis ($r_t
\times R \times f_a$). The balance between these two rates will determine the
effective amino acid concentration in the cell $[AA]_\text{eff}$. An important
premise for this formulation is growing evidence that cells are able to modulate
their biosynthesis activity according to nutrient availability (i.e. extent of
chromosomal replication, transcriptional, and translation activity) through
secondary-messenger molecules like (p)ppGpp \citep{hauryliuk2015, zhu2019,
kraemer2019, fernandezcoll2020, Buke2020}. Given our observation that protein
synthesis and energy production are not limiting, we assume that other molecular
players required by ribosomes like elongation factors and GTP are available in
sufficient abundance. In addition, experimentally, the relative number of tRNA
and elongation factor EF-Tu per ribosome have been found to increase in poorer
nutrient conditions \cite{pedersen1978, dong1996, klumpp2013}).

We begin by considering a coarse-grained description of peptide elongation,
which includes 1) the time required to find and bind each correct amino-acyl
tRNA, and 2) the remaining steps in peptide elongation that will not depend on
the amino acid availability. These time scales will be related to the inverse of the
elongation rate $r_t$,

\begin{equation}
\frac{1}{r_t} = \frac{1}{k_{on} \alpha [AA]_{\text{eff}}} + \frac{1}{r_{t}^{\text{max}}}.
\end{equation}
where we have assumed that the rate of binding by amino-acyl tRNA $k_{on}$ is
proportional to $[AA]_{\text{eff}}$ by a constant $\alpha$. $r_{t}^{\text{max}}$
refers to the maximum elongation rate. This leads to a Michaelis-Menten
dependence of the elongation rate $r_t$ on the effective amino acid
concentration $[AA]_{\text{eff}}$ \citep{klumpp2013, dai2016}.
We can re-write this more succinctly in terms of an effective dissociation
constant,

\begin{equation}
    K_D = \frac{r_{t}^{\text{max}}}{\alpha k_\text{on}},
\end{equation}
where the elongation rate $r_t$ is now given by

\begin{equation}
r_t = \frac{r_{t}^{\text{max}}}{1 + K_D/[AA]_{\text{eff}}}.
\label{eq:rt_kd_simple}
\end{equation}

The rate of amino acid supply $r_{AA}$ will vary with changing nutrient
conditions and the cell can maintain $[AA]_{\text{eff}}$ by tuning the rate of
amino acid consumption, $r_t \times R \times f_a$.  Thus, $[AA]_{\text{eff}}$ is
determined by the difference in the rate of amino acid synthesis (or import, for
rich media) and/or tRNA charging,  $r_{AA}$, and the rate of consumption,
$r_t\times R \times f_a$. Over an  arbitrary length of time $t$ of cellular
growth, the cell will grow in volume, requiring us to consider these rates in
terms of concentration rather than absolute numbers, with $[AA]_{\text{eff}}$
given by,

\begin{equation}
\int_{0}^{t} \frac{d[AA]_{\text{eff}}}{dt} dt =  \int_{0}^{t}([r_{AA}] - [r_t\times R \times f_a]) dt.
\label{eq:aaeff_int}
\end{equation}
This considers the net change in amino acid concentration over a time from 0 to
$t$, with the square brackets indicating concentrations per unit time.
Integrating Equation \ref{eq:aaeff_int} yields.
\begin{equation}
[AA]_{\text{eff}} =  t([r_{AA}] - [r_t \times R \times f_a]).
\label{eq:aaeff_concs}
\end{equation}

Alternatively, to connect to the experimental data in terms of absolute ribosome
copy number $R$ we can consider a unit volume $V$,
\begin{equation}
   [AA]_\text{eff} = \frac{t(r_{AA} - r_t \times R \times f_a)}{V \times N_A},
   \label{eq:aa_final}
\end{equation}
where $r_{AA}$ is in units of AA per unit time and $r_t$ is in units of AA per
unit time per ribosome. $N_A$ refers to Avogadro's number and is needed to
convert between  concentration and absolute numbers per cell. With an expression
for $[AA]_\text{eff}$ in hand, we can now solve Equation \ref{eq:rt_kd_simple} for $r_t$
which is a quadratic function with a physically-meaningful root of

\begin{equation}
r_t = \frac{t(r_{AA} + r_t^\text{(max)}Rf_a) + K_D V N_A - \sqrt{(r_{AA}t + r_t^\text{(max)}Rf_at + K_D V N_A)^2 - 4(Rf_at)(r_t^\text{(max)}r_{AA} t)}}{2Rf_at}.
\label{eq:rt_root}
\end{equation}
This is the key equation that allows us to calculate growth rate for any
combination of $N_\text{pep}$, $R$, $f_a$, and cell size $V$ as a function of
amino acid supply $r_{AA}$ ( \textbf{\textit{Equation 3}}  of the main text). We refer the
reader to Section "A Minimal Model of Nutrient-Mediated Growth Rate Control " of the main text for our exploration of this
model in the context of the proteomic data.

We end this section by noting several distinctions of this formulation with
previous work. The first, as noted in the main text, relates to the now seminal
work of \cite{scott2010}, which provides a treatment of resource allocation that
partitions of the proteome into sectors -- including one for ribosome-associated
proteins and one for metabolic proteins. As cells grow faster, there is a
notable change in the mass fraction of these sectors, with an increase in
ribosomal content that is predominantly achieved at the expense of a decrease in
the metabolic sector. By including an additional constraint through the
phenomenological parameter $\nu$,  which characterizes the quality of the growth
medium \cite{scott2010, klumpp2013, klumpp2014}, the authors derive a model of
growth rate, dependent on optimal resource allocation. Here we have developed a
model that considers the effect of changes in absolute protein abundance and
ribosomal content, and consider how these influence the achievable growth rate.
In addition, by accounting for the metabolic supply of amino acids directly
though their availability in the cell (i.e. $[AA]_\text{eff}$), we are able to
consider how the balance between translation-specific metabolic capacity and
translational capacity influences both the elongation rate $r_t$ and growth
rate $\lambda$.

The second and last point we note is that the recent works from \cite{dai2016}
and \cite{klumpp2013} also employ a similar coarse-graining of translation
elongation as we've considered above. Here, however, a notable distinction is
that the authors consider the entire ternary complex (i.e. the complex of
amino-acyl tRNA, EF-Tu, and GTP) as rate limiting. Further, through an assumed
proportionality between ternary complex and ribosome abundance, they arrive at a
formulation of elongation rate $r_t$ that exhibits a Michaelis-Menten
dependence on the ribosomal fraction $\Phi_R$. They demonstrate that all their
measurements of elongation rate, even upon addition of sublethal doses of
chloramphenicol (which cause an increase in both $r_t$ and $\Phi_R$), can be
collapsed onto a single curve described by this Michaelis-Menten dependence.
There is always a benefit to increase their ribosomal fraction $\Phi_R$ on
growth rate when nutrient conditions allow (see Section "Maximum Growth Rate is Determined by the Ribosomal Mass Fraction" on the main text), and this trend
in the data in part follows from the tendency for cells to increase $\Phi_R$ and
better maximize $r_t$ as nutrient conditions improve.  In addition, it does not
account for the decrease in the fraction of actively translating ribosome $f_a$
that was strikingly apparent at slow growth rates or in sublethal doses of
chloramphenicol in the work of \cite{dai2016}. Through Equation \ref{eq:rt_root} we also
account for changes in the fraction of actively translating ribosomes.
Ultimately, we find that cells are able to maximize both $\Phi_R$, $r_t$, and
their growth rate  only to the extent allowed by the nutrient conditions (i.e.
via $r_{AA}$) and through the maintenance of the cellular pool of amino acids
$[AA]_\text{eff}$, amino-acyl tRNA, GTP, as well as the synthesis of other key
molecular constituents like EF-Tu.
