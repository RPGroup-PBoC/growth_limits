\section{Cell Envelope Biogenesis}
In contrast to nutrient transporters, which support the synthesis of
biomolecules throughout the cell and therefore need to scale with the cell size,
here we must consider the synthesis of components that will need to scale with
the surface area of the cell. \textit{E. coli} is a rod-shaped bacterium with a
remarkably robust length-to-width aspect ratio of $\approx$ 4:1
\citep{harris2018, ojkic2019}. In this section, we consider the synthesis of
lipids as well as the complexes involved in assembling the peptidoglycan
scaffold that makes up the cell envelope.

% Assuming this surface area is approximately the same between the inner and outer
% membranes of \textit{E. coli}, and the fact that each membrane is itself a lipid
% bilayer (or, a bilayer with lipopolysaccharides decorating the outer membrane),
% our rule-of-thumb of 5 \textmu m$^2$ per surface suggests a total membrane
% surface area of $\approx 20  $\textmu m$^2$ (see Appendix Section "Estimation of
% Cell Size and Surface Area" for a description of the calculation of cell surface
% area as a function of cell size). With only $\approx$ 40 \% of the membrane area
% occupied by lipids or lipopolysaccharide, both of which have fatty acid chains
% of similar length (BNID: 100078). Using a rule-of-thumb of 0.5 nm$^2$ as the
% surface area of the typical lipid (BNID: 106993), we can estimate $\approx 2
% \times 10^7$ lipids per cell, which is in close agreement with experimental
% measurements (BNID: 100071, 102996).

The membranes of \textit{E. coli} are composed of a variety of different lipids,
each of which are unique in their structures and biosynthetic pathways
\citep{sohlenkamp2016}. Recently, a combination of stochastic kinetic modeling
\citep{ruppe2018} and \textit{in vitro} kinetic measurements
\citep{ranganathan2012, yu2011} has revealed remarkably slow steps in the fatty
acid synthesis pathways which may serve as the rate limiting reactions for
making new membrane fatty acids (that become components of a variety of
membrane lipids) in \textit{E. coli}. One such step is the removal of hydroxyl
groups from the fatty-acid chain by ACP dehydratase that leads to the formation
of carbon-carbon double bonds. This reaction, catalyzed by proteins FabZ and
FabA \citep{yu2011}, has been estimated to have kinetic
turnover rates of $\approx$ 1 dehydration per second per enzyme
\citep{ruppe2018}. Thus, given this rate and the need to synthesize $\approx$ 2
$\times$ 10$^7$ lipids over 5000 seconds, one can estimate that a typical cell
requires $\approx$ 4000 ACP dehydratases. This is in reasonable agreement with
the experimentally observed copy numbers of FabZ and FabA
[\FIG{nutrient_cellwall_energy}(D)].
% Furthermore, we can extend this estimate to account
% for the change in membrane surface area as a function of the growth rate (grey
% line in \FIG{cell_envelope}(A)), which in contrast to our observations with glucose uptake, indeed captures the observed growth rate
% dependent expression of these two enzymes.

The exquisite control of bacteria over their cell shape is due primarily to a
stiff, several nanometer thick meshwork of polymerized disaccharides that makes
up the cell wall termed the peptidoglycan. The formation of the peptidoglycan is
an intricate process involving many macromolecular players \citep{shi2018,
morgenstein2015}, whose coordinated action synthesizes the individual subunits
and integrates them into the peptidoglycan network that maintains cell shape and
integrity even in the face of large-scale chemical and osmotic perturbations
\citep{harris2018,shi2018}. Due to the extensive degree of chemical crosslinks
between glycan strands, the entire peptidoglycan is a single molecule comprising
$\approx$ 3\% of the cellular dry mass (BNID: 1019360), making it the most
massive molecule in \textit{E. coli}. The polymerized unit of the peptidoglycan
is a N-acetylglucosamine and N-acetylmuramic acid disaccharide, of which the
former is functionalized with a short pentapeptide. With a mass of $\approx$
1000 Da, this unit, which we refer to as a murein subunit, is polymerized to
form long strands in the periplasm which are then attached to each other via
their peptide linkers. Together, these quantities provide an estimate of
$\approx$ 5 $\times$ 10$^6$ murein subunits per cell.

There are various steps which one could consider \textit{a priori} to be a
limiting process in the synthesis of peptidoglycan, including the biosynthesis
steps that occur in the cytoplasm, the transglycosylation reaction which adds
new subunits to the glycan strands, and the formation of the peptide crosslinks
between strands \citep{shi2018,morgenstein2015,lovering2012,barreteau2008}.
Despite the extensive mechanistic characterization of these components,
\textit{quantitative} characterization of the individual reaction rates along
the entire kinetic pathway of remain scarce and make identification of any
particularly slow steps difficult. However, recent measurements for the
crosslinking machinery [transpeptidases, \cite{catherwood2020}] of the
peptidodglycan, which provides lateral structural integrity to the peptidoglycan
shell, have found the turnover of transpeptidases to be rather slow ($\approx$ 2
crosslinking reactions per second). As the primary mechanism of subunit
integration occurs by a complex with both transglycosylation and
transpeptidation activities \citep{shi2018} we therefore consider the
transpeptidation reaction as a reasonable candidate for a rate-limiting step in
growth as it is vital for cell size and shape homeostasis. We estimate that on
the order of $\approx$ 100 transpeptidases per cell are needed for complete
maturation of the peptidoglycan, given a division time of $\approx$ 5000
seconds; a value that is comparable to experimental observations
[\FIG{nutrient_cellwall_energy}(E)]. Expanding this estimate to account for the
changing mass of the peptidoglycan as a function of growth rate [grey line in
\FIG{nutrient_cellwall_energy}(E)] predicts an order-of-magnitude increase in
the abundance of the transpeptidases when the grow rate is increased by a factor
of four. Here, however, the measured complex abundances across the different
proteomic data sets show systematic disagreements and obfuscates any significant
dependence on growth rate.

% In principle, each murein subunit can be involved in such a crosslink. In
% some microbes, such as in Gram-positive bacterium \textit{Staphylococcus
% aureus}, the extent of crosslinking can be large with $>$ 90\% of
% pentapeptides forming a connection between glycan strands. In \textit{E.
% coli}, however, a much smaller proportion ($\approx$ 20\%) of the peptides
% are crosslinked, resulting in a weaker and more porous cell wall
% \citep{vollmer2008a, rogers1980}. The formation of these crosslinks occurs
% primarily during the polymerization of the murein subunits and is facilitated
% by a family of transpeptidase enzymes. The four primary transpeptidases of
% \textit{E. coli} have only recently been quantitatively characterized
% \textit{in vivo}, via liquid chromatography mass spectrometry, which revealed
% a notably slow kinetic turnover rate of $\approx 2$ crosslinking reactions
% formed per second per enzyme as  noted above \citep{catherwood2020}.


While these processes represent only a small portion of
proteins devoted to cell envelope biogenesis, we find it unlikely that they
limit cellular growth in general. The relative amount of mass required for
lipid and peptidoglycan components will decrease at faster growth rates due to a
decrease in the cell's surface area to volume ratio.
Furthermore, despite the slow catalytic rate of fatty acid synthesis and transpeptidation,
there appear to be sufficient protein abundance to support growth.
For FabZ and FabA in lipid
synthesis, experimental data and recent computational modeling has shown that
the rate of fatty-acid synthesis can be drastically increased by increasing
the concentration of FabZ \citep{yu2011, ruppe2018}. With a proteome size of
$\approx$ 3$\times$10$^6$ proteins, a hypothetical 10-fold increase in
expression from 4000 to 40,000 ACP dehydratases would result in a paltry
$\approx$ 1\% increase in the size of the proteome.

% In the context of
% peptidoglycan synthesis, we note that our estimate considers only the
% transpeptidase enzymes that are involved in lateral and longitudinal elongation
% of the peptidoglycan. This neglects the presence of other transpeptidases
% that are present in the periplasm and also involved in remodeling and
% maturation of the peptidoglycan. It is therefore possible that if this was
% limiting growth, the simple expression of more
% transpeptidases would be sufficient to maintain the structural integrity of the
% cell wall.
