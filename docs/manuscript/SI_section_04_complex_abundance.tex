\section{Calculation of Complex Abundance}

All data collected quantified the abundance of individual proteins with high
resolution. After correcting for errors introduced from overestimated volumes
and imposed boundaries on the protein concentration, we are left with a large
data set, largely comparable between one another. However, this work is focused
on estimating the abundance of individual protein \textit{complexes}, rather
than copies of individual proteins. In this section, we outline the procedure we
used to annotate proteins as being part of a macromolecular complex as well as
how we computed their absolute abundance.


Protein complexes, and proteins individually, often have a variety of names,
both longform and shorthand. As individual proteins can have a variety of
different synonyms, we sought to ensure that each protein annotated in the
data sets used the same synonym. To do use, we relied heavily on the EcoCyc
Genomic Database \citep{keseler2017}.  Each protein in available data sets
included an annotation of one of the gene name synonyms as well as an
accession ID -- either a UniProt or Blattner "b-number". We programmatically
matched up individual accession IDs between the proteins in different data sets.
In cases where accession IDs matched but the gene names were different, we
manually verified that the gene product was the same between the datasets and
chose a single synonym.  All code used  in the data cleaning and unification
procedures can be found on the associated
\href{https://github.com/rpgroup-pboc/growth_limits}[GitHub repository]
(DOI:XXX) associated with this paper as well as on the associated
\href{https://rpgroup.caltech.edu/growth_limits}{paper website}.

With each protein product in the data sets conforming to a single identification
scheme, we were tasked to identify the molecular complexes each protein was a
member of. Additionally, we needed to identify how many copies of each protein
were present in each complex (i.e. the subunit copy number) and compute the
estimated abundance complex that accounted for fluctuations in subunit
stoichiometry. To map proteins to complexes, we programmatically accessed the
EcoCyc \textit{E. coli} database \cite{keseler2017} using PathwayTools version
23.0 \cite{karp2019}. With a license for PathWay Tools, we programmatically
mapped each unique protein to its annotated complexes via the
BioCyc Python package. As we mapped each protein with \textit{all} of its
complex annotations, there was redundancy in the dataset. For example, ribosomal
protein L20 (RplT) is annotated to be a component of the 50S ribosome (EcoCyc
complex \texttt{CPLX-03962}) as well as a component of the mature 70S ribosome
(EcoCyc complex \texttt{CPLX-03964}).


In addition to the annotated complex, we collected information of how many
copies of each individual protein is present in each macromolecular complex.
With this number in hand, we calculated the maximum number of complexes that
\textit{could} be formed given the observed abundance of each protein subunit as 
\begin{equation}
    N_\text{complex}^{(\max)}(\text{subunit}) = \frac{N_\text{subunit}^\text{(observed)}}{N_\text{subunit}^\text{(annotated)}}.
    \label{eq:subunit_max}
\end{equation}
For example, the 70S mature ribosome complex has 55 protein components, all of
which are present in a single copy except L4 (RplL), which is present in 4
copies. For each ribosomal protein, we then calculate the  maximum number of
complexes that could be formed using \EQ{subunit_max}. This example, along with
example from 5 other macromolecular complexes, can be seen in
\FIG{complex_counting}.

It is important to note that measurement noise, efficiency of protein
extraction, stochastic errors will mean that the precise value of each
calculation will be different for each component of a given complex. Thus, to
report the total complex abundance, we computed the arithmetic mean of
$N_\text{complex}^\text{(max)}$ for all subunits as 
\begin{equation}
   \langle N_\text{complex} \rangle = \frac{1}{N_\text{subunits}}\sum\limits_i^{N_\text{subunits}} \frac{N_{i}^\text{(observed)}}{N_{i}^\text{(annotated)}}.
   \label{eq:complex_count}
\end{equation}
in \FIG{complex_counting}, we show this mean value as a grey line for a variety
of different complexes. Additionally, we have built an interactive figure
accessible on the \href{https://www.rpgroup.caltech.edu/growth_limits}{paper
website} where the validity of this approach can be examined for any complex
with more than two subunits (thus, excluding monomers and dimers). 

\begin{figure}
    \begin{fullwidth}
        \centering{
            \includegraphics{SI_figs/figSX_subunit_counting.pdf}
            \caption{\textbf{Calculation of the mean complex abundance from
            measurements of single subunits.} Six of the largest complexes (by
            number of subunits) in \textit{E. coli}. Points correspond to the
            maximum number of complexes that can be formed given measurement of
            that individual protein. Solid grey line corresponds to the
            arithmetic mean across all subunits. These data correspond to
            measurements from \cite{schmidt2016} in a glucose-supplemented
            minimal growth medium.}
            \label{fig:complex_counting}
        }
    \end{fullwidth}
\end{figure}

