\section{Additional Analysis Related to Nutrient-Limited Growth Model.}

\section{Calculation of active ribosomal fraction.}

In the main text we estimate the fraction $f_a$ of actively translating ribosomes across
the different datasets available based on the growth-rate dependent measurements from the work
of \citep{dai2016}. Here we provide additional details on how $f_a$ was initially determined,
and how we have used it to estimate the active ribosomal fraction for each data set.

In the work of \citep{dai2016}, the authors independently measured the
translation rate, ribosomal abundance (via the total RNA-to-protein ratio), and
growth rate $\lambda$ across a vast range of growth conditions (growth rates spanning ~ 0
- 2 h$^{-1}$). By requirements of mass balance, and an assumption that cells are doubling their
proteome with each cell division under steady-state growth, we expect,

\begin{equation}
  r_t \cdot R  \lambda \cdot N_{aa}.
\end{equation}
$r_t$ is the translation elongation rate, $R$ is the number of ribosomes, and $N_{aa}$ is the number of peptide bonds that must be formed to double the cell's protein mass. An important observation from the work of \citep{dai2016} was that their measured translation rates and ribosomal abundance were incompatible with this expectation. This was particularly true at slow growth (below about 0.7 h$^{-1}$). The explanation arrived at by the authors is that cells are regulating the fraction of ribosomes that are translating. As further support for this idea, sublethal concentrations of chloramphenicol caused a further decrease in the apparent
fraction of actively translating ribosomes.

In Figure X we show the reported values of $f_a$ as a function of growth rate

in order to maintain


 \cdot R \cdot f_a$ $N_{aa}This corresponds to Equation 3 in the main text, where the

estimate the fraction
