\section{Extending Estimates to a Continuum of Growth Rates}
\label{sec:SI_continuum_est}
In the main text, we considered a standard stopwatch of 5000 s to estimate the
abundance of the various protein complexes considered. In addition to point
estimates, we also showed the estimate as a function of growth rate as
transparent grey curves. In this section, we elaborate on this continuum
estimate, giving examples of estimates that scale with either cell volume, cell
surface area, or number of origins of replication.

\subsection{Estimation of the total cell mass}
For many of the processes estimated in the main text we relied on a cellular dry
mass of $\approx 300$ fg from which we computed elemental and protein fractions
using knowledge of fractional composition of the dry mass. At modest growth
rates, such as the 5000 s doubling time used in the main text, this is a
reasonable number to use as the typical cell mass is $\approx$ 1 pg and
\textit{E. coli} cells can approximated as 70\% water by volume. However, as we
have shown in the preceding sections, the cell size is highly dependent on the growth rate. This means that a dry mass of 300
fg cannot be used reliably across all growth rates.

Rather, using the phenomenological  description of cell volume scaling
exponentially with growth rate, and using a rule-of-thumb of a cell buoyant
density of $\approx 1.1$ pg / fL (BNID: 103875), we can calculate the cell dry mass across a
range of physiological growth rates as
\begin{equation}
    m_\text{cell} \approx \rho V(\lambda) \approx \rho ae^{\lambda * b}
    \label{eq:def_mcell}
\end{equation}
where $a$ and $b$ are constants with units of \textmu m$^3$  and hr,
respectively. The value of these constants can be estimated from the careful
volume measurements performed by \cite{si2017, si2017}, as considered in Appendix \nameref{sec:protein_size_SV} earlier.

\subsection{Complex Abundance Scaling With Cell Volume}
Several of the estimates performed in the main text are implicitly dependent
on the cell volume. This includes processes such as ATP utilization and, most
prominently, the transport of nutrients, whose demand will be proportional to the volume
of the cell. Of the latter, we estimated the
number of transporters that would be needed to shuttle enough carbon,
phosphorus, and sulfur across the membrane to build new cell mass. To do so,
we used elemental composition measurements combined with a 300 fg cell dry
mass to make the point estimate. As we now have a means to estimate the total
cell mass as a function of volume, we can generalize these estimates across
growth rates.

Rather than discussing the particular details of each transport system, we will
derive this scaling expression in very general terms. Consider that we wish to
estimate the number of transporters for some substance $X$, which has been
measured to be made up some fraction of the dry mass, $\theta_X$. If we assume
that, irrespective of growth rate, the cell dry mass is relatively constant
\citep{basan2015} and $\approx$ 30\% of the total cell mass, we can state that
the total mass of substance $X$ as a function of growth rate is
\begin{equation}
m_X \approx 0.3 \times \rho V(\lambda) \theta_X,
\label{eq:m_x}
\end{equation}
where we have used $\rho V(\lambda)$ as an estimate of the total cell mass,
defined in \EQ{def_mcell}. To convert this to the number of units $N_X$ of substance
$X$ in the cell, we can use the formula weight $w_X$ of a single unit of $X$ in
conjunction with \EQ{m_x},
\begin{equation}
    N_X \approx \frac{m_X}{w_X}.
    \label{eq:n_x}
\end{equation}

To estimate the number of transporters needed, we make the approximation that
loss of units of $X$ via diffusion through porins or due to the permeability of
the membrane is negligible  and that a single transporter complex can transport
substance $X$ at a rate $r_X$. As this rate $r_X$  is in units of $X$ per time
per transporter, we must provide a time window over which the transport process
can occur. This is related to the cell doubling time $\tau$, which can be
calculated from the the growth rate $\lambda$ as $\tau = \log(2) / \lambda$.
Putting everything together, we arrive at a generalized transport scaling
relation of
\begin{equation}
N_\text{transporters}(\lambda) = \frac{0.3 \times \rho V(\lambda)\theta_X}{w_X r_X \tau}.
\label{eq:transporter_continuum}
\end{equation}

This function is used to draw the continuum estimates for the number of
transporters seen in Figures 2 and 3 as transparent grey curves. Occasionally,
this continuum scaling relationship will not precisely agree with the point
estimate outlined in the main text. This is due to the choice of $\approx$ 300 fg
total dry mass per cell for the point
estimate, whereas we considered more precise values of cell mass in the continuum estimate. We
note, however, that both this scaling relation and the point estimates are meant
to describe the order-of-magnitude observed, and not the predict the exact
values of the abundances.

\EQ{transporter_continuum} is a very general relation for processes where the
cell volume is the "natural variable" of the problem. This means that, as the
cell increases in volume, the requirements for substance $X$ also scale with
volume rather than scaling with surface area, for example. So long as the rate
of the process, the fraction of the dry mass attributable to the substance, and
the formula mass of the substance is known, \EQ{transporter_continuum} can be
used to compute the number of complexes needed. For example, to compute the
number of ATP synthases per cell, \EQ{transporter_continuum} can be slightly
modified to the form
\begin{equation}
    N_\text{ATP synthases}(\lambda) = \frac{0.3 \times \rho V(\lambda)\theta_{protein}N_\text{ATP}}{w_{AA} r_\text{ATP} \tau},
\end{equation}
where we have included the term $N_\text{ATP}$ to account for the number of ATP
equivalents needed per amino acid for translation ($\approx$ 4, BNID: 114971),
and $w_{AA}$ is the average mass of an amino acid. The grey curves in Figure 4
o the main text were made using this type of expression.

\subsection{A Relation for Complex Abundance Scaling With Surface Area}
In our estimation for the number of complexes needed for lipid synthesis and
peptidoglycan maturation, we used a particular estimate for the cell surface
area ($\approx$ 5 \textmu$m$, BNID: 101792) and the fraction of dry mass
attributable to peptidoglycan ($\approx$ 3\%, BNID: 101936). Both of these
values come from glucose-fed \textit{E. coli} in balance growth. As we are
interested in describing the scaling as a function of the growth rate, we must
also consider how these values scale with cell surface area, which is the natural
variable for these types of processes. In the coming paragraphs, we highlight
how we incorporate a condition-dependent surface area into our calculation of
the number of lipids and murein monomers that need to be synthesized and
crosslinked, respectively.

\subsubsection{Number of Lipids}
To compute the number of lipids as a function of growth rate, we make the
assumption that some features, such as the surface area of a single lipid
($A_\text{lipid} \approx$ 0.5 nm$^2$, BNID: 106993) and the total fraction of the membrane
composed of lipids ($\approx$ 40\%, BNID: 100078) are independent of the growth
rate. Using these approximations combined with \EQ{surface_area}, and
recognizing that each membrane is composed of two leaflets, we can
compute the number of lipids as a function of growth rate as

\begin{equation}
    N_\text{lipids}(\lambda) \approx \frac{4\,\text{leaflets} \times 0.4 \times
    \eta\pi\left(\frac{\eta\pi}{4} -
    \frac{\pi}{12}\right)^{-2/3}V(\lambda)^{2/3}}{A_\text{lipid}}
\end{equation}
where $\eta$ is the length-to-width aspect ratio and $V$ is the cell volume.

\subsubsection{Number of Murein Monomers}
In calculation of the number of transpeptidases needed for maturation of the
peptidoglycan, we used an empirical measurement that $\approx$ 3\% of the dry
mass is attributable to peptidoglycan and that a single murien monomer is
$m_\text{murein} \approx$ 1000 Da. While the latter is independent of growth rate, the former is
not. As the peptidoglycan exists as a thin shell with a width of $w \approx 10$
nm encapsulating the cell, one would expect the number of murein monomers scales
with the surface area of this shell. In a similar spirit to our calculation of
the number of lipids, the total number of murein monomers as a function of
growth rate can be calculated as
\begin{equation}
N_\text{murein monomers}(\lambda) \approx \frac{\rho_\text{pg} w \eta\pi\left(\frac{\eta\pi}{4} -
    \frac{\pi}{12}\right)^{-2/3}V(\lambda)^{2/3}}{m_\text{murein}},
\end{equation}
where $\rho_\text{pg}$ is the density of peptidoglycan.


\subsection{Complex Abundance Scaling With Number of Origins, and rRNA Synthesis}
While the majority of our estimates hinge on the total cell volume or surface
area, processes related to the central dogma, namely DNA replication and
synthesis of rRNA, depend on the number of chromosomes present in the cell. As
discussed in the main text, the ability of \textit{E. coli} to parallelize the
replication of its chromosome by having multiple active origins of replication
is critical to synthesize enough rRNA, especially at fast growth
rates. Derived in \cite{si2017} and reproduced in the main text and Appendix \nameref{sec:SI_ori} below, the average number of
origins of replication at a given growth rate can be calculated as
\begin{equation}
\langle\# \text{ori} \rangle \approx 2^{t_\text{cyc} \lambda / \ln 2}
\label{eq:nori}
\end{equation}
where $t_\text{cyc}$ is the total time of replication and division. We can make
the approximation that $t_\text{cyc} \approx$ 70 min, which is the  time it
takes two replisomes to copy an entire chromosome.

In the case of rRNA synthesis, the majority of the rRNA operons are surrounding
the origin of replication. Thus, at a given growth rate $\lambda$, the average
dosage of rRNA operons per cell $D_\text{rRNA}$ is
\begin{equation}
D_\text{rRNA}(\lambda) \approx N_\text{rRNA operons} \times 2^{t_{cyc} \lambda / \ln 2}.
\label{eq:rRNA_dosage}
\end{equation}

This makes the approximation that \textit{all} rRNA operons are localized around
the origin. In reality, the operons are some distance away from the origin,
making \EQ{rRNA_dosage} an approximation \citep{dennis2004}.

In the main text, we stated that at a growth rate of 0.5 hr$^{-1}$, there is
$\approx$ 1 chromosome per cell. While a fair approximation, \EQ{nori}
illustrates that is not precisely true, even at slow growth rates. In estimating
the number of RNA polymerases as a function of growth rate, we consider that
regardless of the number of rRNA operons, they are all sufficiently loaded with
RNA polymerase such that each operon produces one rRNA per second. Thus, the
total number of RNA polymerase as a function of the growth rate can be
calculated as
\begin{equation}
    N_\text{RNA polymerase}(\lambda) \approx L_\text{operon}D_\text{rRNA}\rho_\text{RNA polymerase},
\end{equation}
where $L_\text{operon}$ is the total length of an rRNA operon ($\approx$ 4500
bp) and $\rho_\text{RNA polymerase}$ is packing density of RNA polymerase on a
given operon, taken to be 1 RNA polymerase per 80 nucleotides.
