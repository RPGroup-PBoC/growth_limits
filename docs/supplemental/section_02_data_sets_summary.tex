\section{Summary of Final Compiled Data Set.}
\label{sec:SI_data_summary}

[NB: in progress]

% NB: need to make summary figure with 2x2 panels; top row A) reported total protein per cells
% B) protein concentration. bottom row, C) new total protein mass per cell,
% C) new protein concentration.

% NB: It may be helpful to appeal to the 'classic growth laws' early on in this text
% as a rational to guide thinking with expectations about total cell mass, cell volume w.r.t.
% growth rate.
%
% The data sets encompass a wide range of bacterial growth conditions and
% different \textit{e. coli} strains. In order to determine protein abundance on a
% cell basis,  different strategies  were taken to determine cell counts and cell
% volume. It was therefore important to consider if any obvious discrepancies
% exist across the data and whether these might be reasonably dealt with to make
% the compiled data set internally consistent. \textit{a priori}, there are
% well-documented observations about how characteristics such as total protein
% mass per cell and cell volume  should scale with growth rate. We were therefore
% inclined to only renormalize data in a  way  that was consistent with
% expectations.
% In this section we describe the data manipulations that were applied to those originally reported.
%

%
% Figure [](A) compares the total protein mass reported as a function of
% growth rate for each of the original data sets, while in Figure \ref[](B)  we
% plot the final values that were used in this work. A .csv file containing
% protein copy numbers per cell and mass per cell across all data sets is available for download on our
% GitHub repository.
