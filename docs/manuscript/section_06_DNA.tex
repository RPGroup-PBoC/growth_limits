\section{Function of the Central Dogma}
Up to this point in the work, we have considered a variety of transport and
biosynthetic processes that must be carried out before the cell can successfully
divide. While there are of course many other metabolic processes we could
consider and perform estimates of (such as the components of fermentative versus
aerobic respiration), we now turn our focus to some of the most central
processes which \textit{must} be undertaken irrespective of the growth
conditions -- the processes of the central dogma. 

\subsection{DNA Replication}
Like most bacteria, the genome of \textit{E. coli} consists of a single,
circular chromosome. Harboring $\approx$ 5000 genes and $\approx 5\times
10^6$ base pairs, this chromosome must be faithfully replicated and
segregated into each nascent cell through a division. We again rely on the
near century of literature in molecular biology to provide some insight
towards the rates and mechanics of the replicative feat. Replication of the
bacterial chromosome is initiated at a single region of the chromosome termed
the \textit{oriC} locus at which a pair of DNA polymerases bind and begin
their high-fidelity replication of the genome in opposite directions.
Assuming equivalence between the two replication forks, this means that the
two DNA polymerases meet at the midway point of the circular chromosome
termed the \textit{ter} locus. This divison of labor means The kinetics of
the four types of DNA polymerases (I -- V) have been intensely studied,
revealing that DNA polymerase III performs the high fidelity processive
replication of the genome with the other "accessory" polymerases playing
auxiliary roles \cite{fijalkowska2012}. \textit{In vitro} measurements have
shown that DNA Polymerase III copies DNA at a rate of $\approx 600$
nucleotides per second (BNID: 104120, \cite{milo2010}). Thus, to replicate a
single chromosome, two DNA polymerases replicating at their maximal rate would
copy the entire genome in $\approx$ 4000 s. Thus, with a division time of 6000
s (our "typical" growthrate for the purposes of this work), there is sufficient
time for a pair of DNA polymerases to replicate the entire genome. However, this
estimate implies that 4000 s would be the upper-limit time scale for bacterial
division which is at odds with the familiar $\approx$ 1500 s doubling time of
\textit{E. coli} in rich medium.

It is know well known that \textit{E. coli} can parallelize its DNA replication
such that multiple chromosomes are being replicated at once. Recent work
\citep{si2017} has shown that the replicative timescale of cell division can be
massively parallelized where \textit{E. coli} can have on the order of 10 - 12
replication forks at a given time.  


[GC: Not sure how much time we should really pay DNA replication as a potential
limiting process. This estimate is a bit strange and has its fair share of
caveats. We can make the argument that using a stopwatch of around 6000 s for a
division, there should be enough time for E. coli to copy its chromosome with a
single pair of DNA polymerases. However, at 1500 s (fast growth), a cell needs
on the order of ~20 assuming 2 per replication fork. This is approximately what
we see in the data, but is not much of an estimate. It's possible we can neglect
replication completely and specifically talk about just dNTP synthesis.]


\subsection{dNTP Synthesis}