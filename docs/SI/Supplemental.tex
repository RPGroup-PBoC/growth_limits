\documentclass[11pt]{article}
\setlength{\topmargin}{-0.25in}
\setlength{\textheight}{8.75in}
\setlength{\oddsidemargin}{.125in}
\setlength{\textwidth}{6.25in}
\linespread{1.4}
\usepackage{graphicx}
\usepackage{chemmacros}
\usepackage{amsmath}
\usepackage{float}
\usepackage{gensymb}
\usepackage{multirow}
\usepackage{color}
\usepackage{wrapfig}
\usepackage[font=small,labelfont=bf]{caption}
%\usepackage{epsfig}
\author{N. Belliveau, G. Chure, J. Theriot, R. Phillips}

\begin{document}
\title{Supplemental Information}
\maketitle

\section{Summary of Proteome Datasets.}

Here we briefly summarize the datasets that were used for the work of the
main text. The goal of this section is to give an overview of each
dataset considered, including the main experimental details, and to provide a
more detailed look at how well each compares.

Table \ref{table:datasets} provides an overview of the datasets that were considered,
which are predominately mass spectrometry based, with the exception of the work from Li et al. (2014)
using ribosomal profiling, and the fluorescence-based counting done in Taniguchi et al. (2010).


\begin{center}
\begin{tabular}{ || c | c | c || }
\hline
Author & Method & Reported Quantity \\
\hline\hline
Taniguchi et al. (2010) & YFP-fusion, cell fluorescence & fg per cell \\
\hline
Valgepea et al. (2012) & Mass spectrometry & fg per cell \\
\hline
Peebo et al. (2014) & Mass spectrometry & fg per fL \\
\hline
Li et al. (2014) & Ribosomal profiling & protein synthesis rate \\
\hline
Soufi et al. (2015) & Mass spectrometry & fg per cell \\
\hline
Schmidt et al. (2016) & Mass spectrometry & fg per cell \\
\hline
Caglar et al. (2017) & Mass spectrometry & relative abundance \\
\hline
\end{tabular}
\label{table:datasets}
\end{center}

\section{Adjustments to Copy Number Data.}

%
% \begin{figure}[H]
% 		\centering
%   \begin{tabular}{ l  l }
% 	  A) & B) \\
% 	      \includegraphics[width=0.55\textwidth]{SILAC_corr_O3_2.pdf} &
% 	       \includegraphics[width=0.55\textwidth]{deltaSILAC_corr_O3.pdf} \\
%   \end{tabular}
%   \caption{}
%   \label{fig:HQ1Q2}
% \end{figure}


\end{document}
