\section{Estimation of total protein, cell size, and surface area across all growth conditions.}

In \FIG{cell_size_literature} we looked at a number of recent cell
size measurements and potential issues with the values used by Schmidt
\textit{et al.}. Since most of the proteomic data sets lack cell size
measurements, we chose instead to use a common set of size measurements for any
analysis requiring cell size or surface area.  Since each of the data sets used
either K-12 MG1655 or its derivative, BW25113 (from the lab of Barry L. Wanner;
the parent strain of the Keio collection \citep{datsenko2000, baba2006}), we fit
the MG1655 cell size data from Si \textit{et al.} 2017, 2019 using
the optimize.curve\_fit function from the Scipy python package \citep{2020scipynmeth}.

The size data is shown in Figure \FIG{final_size_data_Si}(A)
and (B),  for the cell length and width, respectively. The length data was well
described by the exponential function 0.5 $e^{1.09 \cdot \lambda}$ + 1.76
\textmu m, while the width data was well described by 0.64 $e^{0.24 \cdot \lambda}$
\textmu m. In order to estimate cell size we take the cell as a cylinders
with two hemispherical ends \citep{si2017, basan2015}. Specifically,  cell size
(or volume) is estimated from,

\begin{equation}
V = \pi \cdot r^2 \cdot (l - 2r/3),
\label{eq:cell_size}
\end{equation}
where $r$ is half the cell width. A best fit to the data is described by 0.533
$e^{1.037 \cdot \lambda}$ \textmu m$^3$. Calculation of the cell surface area is
given by,

\begin{equation}
 S = \eta \cdot \pi (\frac{\eta \cdot \pi}{4} - \frac{\pi}{12})^{-2/3} V^{2/3},
\end{equation}
where $\eta$ is the aspect ratio ($\eta$ = $l/w$) \citep{ojkic2019}.

\begin{figure}
		\centering
    \includegraphics[width=1.0\textwidth]{SI_figs/Si_size_data_fit.pdf}
    \caption{\textbf{Summary of size measurements from Si \textit{et al.} 2017,
    2019.} Cell lengths and widths were measured from cell contours obtained from
    phase contrast images, and refer to the long and short axis respectively. (A)
    Cell lengths and (B) cell widths show the mean measurements reported (they
    report 140-300 images and 5,000-30,000 for each set of samples; which likely
    means about 1,000-5,000 measurements per mean value reported here since they
    considered about 6 conditions at a time). Fits were made to the  MG1655 strain
    data; length: 0.5 $e^{1.09 \cdot \lambda}$ + 1.76 \textmu m, width:  0.64
    $e^{0.24 \cdot \lambda}$ \textmu m. (C) Cell size, $V$, was calculated as
    cylinders with two hemispherical ends (Equation \ref{eq:cell_size}). The
    MG1655 strain data gave a best fit of 0.533 $e^{1.037 \cdot \lambda}$ \textmu m$^3$.}
  \label{fig:final_size_data_Si}
\end{figure}
