
% This overabundance of ribosomes
% provides different challenges on the ability of the cell to maintain steady-state
% growth under limiting nutrient conditions, and in Supplemental
% Section XX we consider this slow growth regime further.

% While this dependence between cell size and ribosomal abundance is apparent
% across moderate to fast growth rates, it is worth noting that

% This scaling behavior is likely to change at slow growth rates (below $\lambda
% \approx 0.5\,\text{hr}^{-1}$). Here, the number of ribosomes $R$ no longer
% reflects the cell's protein synthesis capacity, so far taken to be $r_t \ R$.
% Instead, cells reduce the number of actively translating ribosomes through the
% additional regulatory control of the small-molecule alarmone, guanosine
% pentaphosphate [(p)ppGpp] \citep{dai2016} [more citations].

\subsection{[Header, Fill me in???]}

The dependence between cell size, actively translating ribosomes, and growth
rate, suggest the possibility that cells tune their size and ribosomal abundance
to match their biosynthetic capacity to the available nutrient conditions. As
one illustration of this, disruption of the major glucose uptake system PTS
through deletion ptsG doesn't just simply limit carbon uptake and therefore growth,
which is reduced about two-fold. Rather, cells also adjust to this perturbation
by reducing their ribosomal fraction about two-fold \citep{dai2016}, matching
the expectation from \EQ{translation_limit_growth_rate} if cells are growing
near their maximal translation capacity.

Under changes of stress and nutrient limitation, bacteria rely on
secondary-messenger molecules to quickly respond to and adjust to changes in
their environment. In \textit{E. coli}, response to changes in amino acid
availability is largely mediated through the small-molecule alarmones, guanosine
pentaphosphate [(p)ppGpp] \citep{dai2016} [more citations]. Amino acid
starvation causes the accumulation of deacylated tRNAs at the ribosome's A-site
and leads to a strong activation of (p)ppGpp synthesis activity by RelA as part
of the stringent response \citep{hauryliuk2015}.



there
are examples where

In loose terms , show that cells are


The translation-limited growth rate (\EQ{translation_limit_growth_rate})
highlights the necessity for a cell to increase its ribosomal fraction to
maximize its growth rate. However, it ignores the absolute constraints of a
growing cell, which in particular, require resources to support each translating
ribosomes and the space to house them. For almost every one of our estimates,
the proteomic data suggest that cells predominantly vary their protein
abundances in order to keep pace with the demands of a growing cell. In this
final section we consider a minimal model of growth rate control. We use it to
provide intuition into the additional constraints on that arise if cells did not
tune their ribosomal abundance and cell size according to the available nutrient
conditions [ref figure (A)].

% As a final comment, it has recently been shown that growth in a (p)ppGpp null
% strain also lacked both the condition-dependent changes in $\langle$\#
% ori$\rangle$ as well as changes in cell size across different growth condition.
% Instead, cells always exhibited a high ratio of $\langle$\# ori$\rangle$ to
% $\langle$\# ter$\rangle$, irrespective of growth rate, and a cell size that was
% more consistent with a fast growth state where (p)ppGpp levels are normally low
% \citep{fernandezcoll2020} and ribosomal fraction is high \citep{zhu2019}. There
% is also evidence that this may be achived through inhibition of DNA replication
% initiation \citep{kraemer2019}. These observations raise the possibility that
% (p)ppGpp may be playing a causal role in tuning $\langle$\# ori$\rangle$ and
% cell size, which ultimately allows the cell to vary its ribosomal content
% according to nutrient availability.

[Develop idea that aa/nutrient sensing may be the way cells are tuning their size to
match nutrient condition.]

In order for cells to maximize their rate of protein synthesis, $r_t R$,
ribosomes must be able to rapidly match codons with their correct amino-acyl
tRNAs. At minimum, this requires synthesis of (or import, for rich media) amino
acids that maintain the pool of amino-acyl tRNAs. We therefore consider this
maintenance as simply a balance between the the supply of amino acids by
metabolic proteins and transporters, at a rate of $r_{aa}$ in units of amino
acids per second, and consumption by ribosomes at a rate of $r_t R f_a$.
To proceed we will first determine the
relationship between these parameters and the effective pool of amino acids,
denoted by $[AA]_{\text{eff}}$, and then use this to calculate the rate of
elongation $r_t$.

\subsubsection{A minimal model of nutrient-limited growth}


The rate of elongation $r_t$ will depend on the availability of amino acids
(and, also amino-acyl tRNAs) in the cell. To allow us to consider the effects of
any limiting metabolic supply (i.e. insufficient $r_{aa}$), or excessive
consumption by ribosomes (i.e. high $R f_a$), we treat $[AA]_{\text{eff}}$ as
a potential rate-limiting step during translation. Specifically, we assume that
the rate of elongation $r_t$ depends on two course-grained timescales, 1) the
time to find and bind each correct amino-acyl tRNAs, and 2) the remaining steps
in peptide elongation that will not depend on the amino acid concentration. The
time to translate each codon is given by the inverse of the elongation rate
$r_t$, which can be written as,

\begin{equation}
\frac{1}{r_t} = \frac{1}{k_{on} B [AA]_{\text{eff}}} + \frac{1}{r_{t}^{\text{max}}}.
\end{equation}
Here we have assumed that the rate of binding by amino-acyl tRNA $k_{on}$ will
be proportional to $[AA]_{\text{eff}}$ in limiting conditions [cite?] by some
proportionality constant $B$. The second term on the right-hand side reflects a second
assumption that elongation factors and GTP are in sufficient abundance across
growth conditions and a maximum elongation rate $r_{t}^{\text{max}}$
of about 17 aa per second \cite{dai2016}. This can be rearranged more succinctly in
terms of an effective binding constant $K_d = r_{t}^{\text{max}}/ (k_{on} \ B)$, with
the elongation rate now given by,

\begin{equation}
r_t = \frac{r_{t}^{\text{max}}}{1 + K_d/[AA]_{\text{eff}}}.
\label{eq:rt_kd_simple}
\end{equation}


During steady-state
growth the amino acid concentration is constant ($d[AA]_{\text{eff}}/dt$=0) and the rates of supply
$r_{aa}$ and consumption $r_t R f_a$ are related to $[AA]_{\text{eff}}$ by,

\begin{equation}
\int_{0}^{\tau} \frac{d[AA]_{\text{eff}}}{dt} dt =  \int_{0}^{\tau} ([r_{aa}] - [r_t R f_a]) dt,
\end{equation}
where the time from 0 to $tau$ is a single cell doubling, and the square brackets indicate of concentration per time. Solving this, we find that

\begin{equation}
[AA]_{\text{eff}} =  ([r_{aa}] - [r_t R f_a]) \tau.
\end{equation}
Alternatively, for an average cell size of $V$,  $[r_{aa}] = r_{aa}/(V N_A)$
and $[r_t R f_a] = (r_t R \ f_a)/(V N_A)$, where $N_A$ is Avogadro's
number. Since $\tau = ln(2)/\lambda$, which is also related to the parameters
$r_t R f_a$ and $N_{aa}$ through \EQ{mass_balance}, we
can also rewrite this as,

\begin{equation}
[AA]_{\text{eff}} = \frac{ln(2) N_{aa}}{V N_A} \left(\frac{r_{aa}}{r_t R f_a} - 1 \right) .
\label{eq:aa_final}
\end{equation}



By plugging in \EQ{aa_final} into \EQ{rt_kd_simple}, which also depends on $r_t$, we
can solve for $r_t$ explicitly. Its solution are the roots of a quadratic equation,
with the positive given by,

\begin{equation}
r_t = \frac{\sqrt{c^2 + 4 c k r_{t}^{\text{max}} - 2 c r_{t}^{\text{max}} + (r_{t}^{\text{max}})^2} - c - r_{t}^{\text{max}}}{2 \ (k-1)}.
\label{eq:rt_final}
\end{equation}
Here, $c = r_{aa}/(R f_a)$ and $k = N_A V K_d / N_{aa}$. In the final two
subsections we use this model to explore how the cell's metabolic capacity
($r_{aa}$) constrains the maximum growth rate, and then explain an apparent role
of for (p)ppGpp in mitigating translational activity at slow growth, where the
number of ribosomes is in excess.

\subsubsection{\textit{E. coli} Tune Ribosomal Content to
Maximize Growth According to Nutrient availability}



\subsubsection{\textit{E. coli} Maintains Growth in Poor Nutrient Conditions by a
Reduction of Translation Activity.}

In the slowest growth condition considered in the proteomic data, cells were
grown in a chemostat ($\lambda$ = 0.12 hr$^{-1}$), and were found to have a ribosomal
mass fraction of $\Phi_R \approx 0.06$ and of order 10$^4$ ribosomes per cell.
In this absence of any additional regulation, we would predict a very low elongation rate of $r_t = X aa/s$.

In contrast, wild-type \textit{E. coli} maintain a relatively high
elongation rate even in stationary phase ($\approx$ 8 AA/s, \citep{dai2016,
dai2018}).

Careful


In \FIG{}(B), we see that in the poorest nutrient conditions
[NB: start by highlighting the dilemma when cells have excess ribosomes].

How do cells regulate protein synthesis when amino acids become limiting,
meaning that consumption exceeds the rate of synthesis? In the slowest  growth
conditions, we find a minimum ribosomal mass fraction of $\Phi_R \approx 0.06$
and  of order 10$^4$ ribosomes per cell.  Without the additional regulatory
control noted above, there would be a point where  this imbalance would occur if
all ribosomes were actively translating  (\FIG{translation_ecoli_partB}).

Such a
scenario would prevent continuous growth, and indeed for (p)ppGpp null strains,
cells only grow in minimal media if additional amino acid supplements are
present. In contrast, wild-type \textit{E. coli} maintain a relatively high
elongation rate even in stationary phase ($\approx$ 8 AA/s, \citep{dai2016,
dai2018}).

To better understand how regulation of ribosomes influence growth rate at
slow growth, we consider a coarse-grained model that relates elongation
rate to a limiting supply of amino acids, which for simplicity we treat as a
single, effective rate-limiting species $[AA]_{eff}$. Under such a scenario, the elongation
rate $r_t$ can be described as depending on the maximum elongation rate ($r_t^{max}
\approx$ 17.1 AA/s, \citep{dai2016, dai2018}), an effective binding constant
$K_D$ between the pool of amino acids and their amino-acyl tRNAs, and the limiting
amino acid concentration $[AA]_{eff}$,

\begin{equation}
r_t = r_t^{max} \cdot \frac{1}{1 + K_D / [AA]_{eff}}.
\label{eq:rate_Kd}
\end{equation}
For cells growing in minimal medium supplemented with glucose, the amino acid
concentration is of order 100 mM (BNID: 110093, \citep{milo2010, bennett2009}).
To estimate  $K_D$, we note that for a growth rate of about 0.6 hr$^{-1}$
\cite{dai2016} measured an elongation rate of about 12.5 AA$\cdot$s$^{-1}$,
yielding $K_D \approx 40$ mM. The maintenance of this amino acid pool
$[AA]_{eff}$ will depend on the difference between the synthesis/supply rate of
amino acids $r_{AA}$ and consumption by ribosomes $r_t \times R \times f_a$,
where we use $f_a$ to account for the possible reduction of actively translating
ribosomes (see Supplemental Appendix XX for further details on this model).

In \FIG{translation_ecoli_partB}(B), we show the relationship between the growth
rate and elongation rate as a function of the number of actively translating
ribosomes. Here, growth rate is now determined by the active ribosomal fraction via
\begin{equation}
\lambda_\text{translation-limited} \approx \frac{r_t}{L_R}  \Phi_R f_a.
\label{eq:translation_limit_growth_rate_2}
\end{equation}
If we consider constant values of amino acid synthesis rate $r_{AA}$ (dashed
lines) to reflect the available parameter space for a specific growth condition,
the fastest growth rates result from  maximization of the fraction of actively translating
ribosomes. When we consider the experimental measurements from \cite{dai2018}
(yellow circles), reflecting growth in different nutrient conditions, we see
that although $R \times f_a$ is reduced in poorer nutrient conditions, it is
reduced in a manner such that $[AA]_{eff}$ is relatively constant. Given our estimate
$K_D \approx$ 40 mM,  we would only expect a decrease from 100 mM to about 35 mM
in the slowest growth conditions. While experimental data is scarce, data from
\cite{bennett2009} show that amino acid
concentrations only decrease to about 60 mM for cells grown in minimal media
supplemented with acetate ($\lambda \approx$  0.3 hr$^{-1}$ in our proteomic data)
\citep{bennett2009}, qualitatively consistent with our expectations. One
explanation for the experimental data is that the active fraction of the
ribosome pool is regulated in order to maintain a sufficient supply of amino acids for
growth. Any further increase in $R \times f_a$ at constant $r_{AA}$ would
otherwise be associated with an additional drop in cellular amino acid
concentration.

\begin{figure*}
    \begin{fullwidth}
    \centering{
        \includegraphics{main_figs/fig8_ribosome_growth_limit_ecoli_b_2.pdf}
        \caption{\textbf{\textit{E. coli} must regulate ribosomal activity in
        limiting nutrient conditions. }
        (A) Translation elongation rate is plotted
        as a function of the number of actively translating ribosomes $R
        f_a$. Dashed lines correspond to a range of amino acid synthesis rates
        $r_{aa}$, from 10$^3$ to 10$^6$. Growth rates are calculated according
        to Equation 1, assuming a constant ribosomal fraction of 8 percent. See
        appendix XX for additional details. (B) Experimental data from \citep{dai2016}
        are used to estimate the fraction of actively translating ribosomes. The solid line represents the translation-limited
        growth rate for ribosomes elongating at 17.1 AA/s.}
        \label{fig:translation_ecoli_partB}
    }
    \end{fullwidth}
\end{figure*}


%
% [important point is that as resources become limiting, cells are able to tune and minimize
% the entire cell mass - which enables them to grow faster]
%
% [Discuss implications of findings so far. All other components being tuned (mostly) in the required proportions; and also that the achievable growth rate is ultimately set by ribosomes. ]
%
% [The apparent constraint that cells MUST get larger in order to grow faster places a particular constrain on why a cell would vary its ribosomal fraction in the first place.]
%
% [Also present the evidence from literature that there is a prominent role for aa/nutrient sensing that appears to mediate X, Y, and Z. ]
%
% [Then go on to propose the way to understand what’s going on with our model.
%     - I think maybe it’s easiest to understand it is we start by considering the balance that must be maintained between metabolic activity and ribosomal activity.
%     - Then write down a model that relates how elongation rate will relate to this.]
%
%
%
% gives no intuition into the additional constraints of a growing
% cell that we've consider throughout our estimates.
%
%
%
% quantifies the maximum rate of growth but ignores the additional constraint set  that there is a maximum rate us with a maximum
%
% [Figure idea: PART (A) Maybe we can consider that r_t*R sets the maximum rate of growth; but you also need to feed that engine and build the house to fit them.]
%
% [ When I get into slow growth regime; consider that IF R scales with <#ori>, ribosomes will have increasingly longer wait times for protein synthesis. Which is bad.
%     - By also decreasing the fraction of actively translating ribosomes, cells can grow faster is also super important. ]
%
%
%
%
