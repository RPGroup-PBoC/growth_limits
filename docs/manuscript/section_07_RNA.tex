\subsection{RNA Synthesis}
With the machinery governing the replication of the genome accounted for, we
now turn our attention to the next stage of the central dogma -- the
transcription of DNA to form RNA. We primarily consider three major groupings
of RNA, namely the RNA associated with ribosomes (rRNA), the RNA encoding the
amino-acid sequence of proteins (mRNA), and the RNA which links codon
sequence to amino-acid identity during translation (tRNA). Despite the varied
function of these RNA species, they share a commonality in that they are
transcribed from DNA via the action of RNA polymerase. In the coming
paragraphs, we will consider the synthesis of RNA as a rate limiting step in
bacterial division by estimating how many RNA polymerases must be present to
synthesize all necessary rRNA, mRNA, and tRNA.

\subsubsection{rRNA}
We begin with an estimation of the number of RNA polymerases needed to
synthesize the rRNA that serve as catalytic and structural elements of the
ribosome. Each ribosome contains three rRNA molecules of lengths 120, 1542,
and 2904 nucleotides (BNID: 108093, \cite{milo2010}). Thus, each ribosome
consists of $\approx$ 4500 nucleotides. The \textit{E. coli} RNA polymerase
transcribes DNA to RNA at a rate of $\approx$ 40 nucleotides per second
(BNID: 101904, \cite{milo2010}). Thus, it takes a single RNA polymerase
$\approx$ 100 s to synthesize the RNA needed to form a functional ribosome.
Therefore, in a 6000 s division time, a single RNA polymerase transcribing
rRNA at a time would result in only $\approx$ 60 functional ribosomal rRNA
units -- far below the observed number of $\approx 10^4$ ribosomes per cell.

Of course, there can be more than one RNA polymerase transcribing at any
given time. To elucidate the \textit{maximum} number of rRNA units that can
be synthesize given a single copy of each rRNA gene, we will consider a
hypothesis in which the rRNA operon is completely tiled with RNA polymerase.
How many polymerase could in principle fit on the rRNA operon? \textit{In
vivo} measurements of the kinetics of rRNA transcription have revealed that
RNA polymerase are loaded onto the promoter of an rRNA gene at a rate of
$\approx$ 1 per second (BNID: 111997; 102362, \cite{milo2010}). If RNA
polymerases are being constantly loaded on to the rRNA genes at this rate,
then we can make the approximation that $\approx$ 1 functional rRNA unit is
synthesized per second. With a 6000 second division time, this hypothesis
leads to a maximal value of 6000 functional rRNA units, still undershooting
the observed number of $10^4$ ribosomes per cell.

\textit{E. coli} has evolved a clever mechanism to surpass this kinetic limit
for the rate of rRNA production. Rather than having only one copy of each
rRNA gene, \textit{E. coli} has seven copies of the operon (BIND: 100352,
\cite{milo2010}) all of which are localized near the origin of replication
\citep{birnbaum1971}. As fast growth requires that multiple copies are being
synthesized simultaneously, this means that the total number of rRNA genes
can be be on the order of $\approx$ 10 -- 30 at a given time
\citep{stevenson2004}. Using our standard time scale of a 6000 second
division time, we can make the lower-bound estimate that the typical cell
will have 7 copies of the rRNA operon. Synthesizing one functional rRNA unit
per second per operon, a total of $4 \times 10^4$ rRNA units can be
synthesized, comfortably above the observed number of ribosomes per cell.

How many RNA polymerases are then needed to constantly transcribe 7 copies of
the rRNA genes? We approach this estimate by considering the maximum number
of RNA polymerases can be tiling the rRNA genes with a loading rate of 1 per
second and a transcription rate of 40 nucleotides per second. Considering
that a RNA polymerase has a physical footprint of approximately 40
nucleotides (BNID: 107873, \cite{milo2010}), we can state that there is
$\approx$ 1 RNA polymerase per 80 nucleotides. With a total length of
$\approx$ 4500 nucleotides per operon and 7 operons per well, the maximum
number of RNA polymerases that can be transcribing rRNA at any given time is
$\approx$ 400, setting a lower bound for the number of RNA polymerase
required to make enough rRNA. As we will see in the coming sections, the
synthesis of rRNA is the dominant requirement of the RNA polymerase pool.

\subsubsection{mRNA}
To form a functional protein, all protein coding genes must first be
transcribed from DNA to form an mRNA molecule. While each protein requires an
mRNA blueprint, many copies of the protein can be synthesized from a single
mRNA. Factors such as strength of the ribosomal binding site, mRNA stability,
and rare codon usage frequency dictate the number of proteins that can be
made from a single mRNA, ranging from 10$^1$ to 10$^4$ (BNID: 104186; 100196;
106254, \cite{milo2010}). Computing the geometric mean of this range yields
$\approx$ 1000 proteins synthesized per mRNA, a value that emerges from
quantitative measurements of the number of proteins per cell ($\approx 3
\times 10^6$, BNID: 100088, \cite{milo2010}) and total number of mRNA per
cell ($\approx 3 \times 10^3$, BNID:100064, \cite{milo2010}). In \textit{E.
coli}, the average protein is $\approx$ 300 amino acids in length (BNID:
108986; \cite{milo2010}), meaning that the corresponding mRNA is $\approx$
900 nucleotides which we will further approximate to be $\approx$ 1000
nucleotides given non-protein coding regions of the mRNA present on the 5'
and 3' ends. With 3000 mRNA per cell, each around 1000 nucleotides in length,
a total of $3 \times 10^6$ nucleotides must be linked together via RNA
polymerase during transcription. With a 6000 second division time and a
typical transcription rate of 40 nucleotides per second per polymerase, we
arrive at a final estimate of $\approx 10$ RNA polymerase complexes are
necessary. This requirement is minuscule compared to the $\approx$ 400
polymerases needed to synthesize the necessary pool of rRNA molecules.

\subsubsection{tRNA}
Our final class of RNA molecules worthy of quantitative consideration is the
the pool of tRNAs used during translation to map codon sequence to amino acid
identity. Unlike mRNA or rRNA, each individual tRNA is remarkably short,
ranging from 70 to 95 nucleotides each (BNID: 109645; 102340,
\cite{milo2010}). What they lack in length, they make up for in abundance.
There are approximately $\approx$ 3000 tRNA molecules present for each of the
20 amino acids (BNID: 105280, \cite{milo2010}), although the precise copy
number is dependent on the identity of the amino acid identity.  Using these
values, we make the estimate that $\approx 5 \times 10^6$ nucleotides are
sequestered in tRNA per cell. Using a similar approach as for our estimate of
mRNA copy number, the cell requires $\approx$ 20 RNA polymerases to polymerase
these nucleotides in a 6000 second time window. This requirement, much like the
requirement for mRNA synthesis, pales in comparison to the number of polymerases
needed to generate the rRNA pool.

\subsubsection{RNA Polymerase and $\sigma$-factor Abundance} 
These estimates, summarized in \FIG{RNA_synthesis} (A), reveal that synthesis
of rRNA is the dominant force dictating the number of RNA polymerases needed
per cell. For completeness, we can use our estimates of $\approx$ 400, 10,
and 20 RNA polymerases needed to synthesize the required number of rRNAs,
mRNAs, and tRNAs, respectively, to state that the typical cell needs to
maintain a pool of $\approx$ 500 RNA polymerases. As is revealed in
\FIG{RNA_synthesis} (B), this estimate ($\approx$ 500) is about and an order
of magnitude below the observed number of RNA polymerase complexes per cell
($\approx$ 5000). 
