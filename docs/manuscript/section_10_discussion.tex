\section{Discussion}

[Fill in.]



A number of recent papers further highlight the possibility that regulation
(p)ppGpp may be the critical component of the apparent scaling laws in
\textit{E. coli}. In the context of ribosomal activity, increased levels of
(p)ppGpp are associated with lower ribsomal content, and at slow growth
are associated with reduced activity by ribosomes and RNA polymerse \citep{dai2016,
dai2018} [citations for RNAP]. Titration of the cellular (p)ppGpp concetrations (up or down) can
invoke similar proteomic changes reminiscent of those observed under nutrient
limitation \citep{zhu2019}. In light of the limiting dependence of ribosome copy
number on chromosomal gene dosage, it was recently shown that growth in a
(p)ppGpp null strain abolishes both the scaling in cell size  and the
$\langle$\# ori$\rangle$ / $\langle$\# ter$\rangle$ ratio. Instead, cells
exhibited a high $\langle$\# ori$\rangle$ / $\langle$\# ter$\rangle$ closer to 4
and cell size more consistent with a fast growth state where (p)ppGpp levels are
low \citep{fernandezcoll2020}. From these results, perhaps the perspective to
take is that the scaling laws  reflect an attempt for the cell to mitigate its
biological activity according  to available nutrients.
