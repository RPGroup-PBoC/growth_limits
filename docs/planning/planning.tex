% Template for PLoS
% Version 1.0 January 2009
%
% To compile to pdf, run:
% latex plos.template
% bibtex plos.template
% latex plos.template
% latex plos.template
% dvipdf plos.template

\documentclass[preprint,10pt]{elsarticle}
% amsmath package, useful for mathematical formulas
\usepackage{amsmath}

\usepackage{graphicx}
\usepackage[version=4]{mhchem}
% amssymb package, useful for mathematical symbols
\usepackage{amssymb}
%\usepackage{booktabs}
%\usepackage[super]{natbib}
% graphicx package, useful for including eps and pdf graphics
% include graphics with the command \includegraphics
%\usepackage{graphicx}

% cite package, to clean up citations in the main text. Do not remove.
%\usepackage{cite}
%\usepackage[numbers,sort&compress]{natbib}

\usepackage{color}

% Use doublespacing - comment out for single spacing
%\usepackage{setspace}
%\doublespacing

%%%%%%%%%%%%%%%%%%%%%%%%%%%%%%%%%%%%%%%%%%%%%%%%%%%%%%%%%%%%%%%%%%%%%%%%

%Added by HG
\let\mpar=\marginpar
\renewcommand\marginpar[1]{\mpar{\raggedright \scriptsize #1}}
\usepackage{booktabs}
\usepackage{ulem}
%This is so I can use the external file from the other parts of the paper.
\usepackage{xr}
\externaldocument{SI/SI-AnnRevRepressionPyramid_Submit}
\usepackage{color}
%%%%%%%%%%%%%%%%%%%%%%%%%%%%%%%%%%%%%%%%%%%%%%%%%%%%%%%%%%%%%%%%%%%%%%%%



% Text layout
\topmargin 0.0cm
\oddsidemargin 0.5cm
\evensidemargin 0.5cm
\textwidth 16cm
\textheight 21cm
% Bold the 'Figure #' in the caption and separate it with a period
% Captions will be left justified
\usepackage[labelfont=bf,labelsep=period,justification=raggedright]{caption}

% Use the PLoS provided bibtex style
% \bibliographystyle{plos2009}

% Remove brackets from numbering in List of References
\makeatletter
\renewcommand{\@biblabel}[1]{\quad#1.}
\makeatother
% Leave date blank
\date{}

\pagestyle{myheadings}
%% ** EDIT HERE **


%% ** EDIT HERE **
%% PLEASE INCLUDE ALL MACROS BELOW
\usepackage{graphicx}
\usepackage{dcolumn}% Align table columns on decimal point
\usepackage{bm}% bold math
\usepackage{hyperref}

\newcommand{\micron}{\mu\text{m}}
\newcommand{\nm}{\text{nm}}
\newcommand{\GC}[1]{\textcolor{blue}{#1}}
\newcommand{\NB}[1]{\textcolor{red}{#1}}

% cite eqations in the main text, with the prefic M:
\usepackage{xr}


%% END MACROS SECTION
\begin{document}

% Title must be 150 characters or less
\begin{flushleft}
{\Large
\textbf{General planning.}
}\\
\end{flushleft}

\section*{Avenues to explore in data.}

\textbold{Biology by the numbers in \textit{E. coli}}:
\begin{enumerate}
  \item General overview/outline of proteome. Are the numbers compatible with prior expectations?
  \item Things I had written down to explore:  growth rate versus volume, Protein density
  \item 'Building E. coli': make use of known fluxes, reaction rates, protein numbers, growth rates -
  do the numbers make sense? Also of interest to see how they vary with growth rate.
  \begin{enumerate}
    \item replication: DNA polymerase; other proteins needed to find origin.
    \item transcription: RNA polymerase, sigma factors
    \item translation: ribosomal proteins, elongation factor, ...
    \item lipid synthesis
    \item nutrient uptake
    \item energy utilization and conversion
  \end{enumerate}
\end{enumerate}

\textbold{Additional insight from  the data}:
\begin{enumerate}
  \item Knowing the unknown: One example here is the high variability in copy number for a
  large fraction of genes that are devoid of any regulatory annotation.
  \item Observation that ratio of transcription factor copy number to (expected) DNA context is
  essentially constant across growth conditions.
\end{enumerate}

\textbold{Quantitative predictions}:
\begin{enumerate}
  \item Essential proteome: What fraction; how does it vary with growth rate?
\end{enumerate}

\section*{What other datasets might be valuable}

There are additional data sets on protein copy number from Gene-Wei Li 2014, Taniguchi 2010.
GC also had another proteomic dataset?. EcoCyc of course. What are other resources?

\section*{Summary of what others have done already.}
\textbold{Papers}:
\begin{enumerate}
  \item 1.
\end{enumerate}

\end{document}
