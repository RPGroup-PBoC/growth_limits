\documentclass[11pt]{article}
\setlength{\topmargin}{-0.25in}
\setlength{\textheight}{8.75in}
\setlength{\oddsidemargin}{.125in}
\setlength{\textwidth}{6.25in}
\linespread{1.4}
\usepackage{graphicx}
\usepackage{chemmacros}
\usepackage{amsmath}
\usepackage{float}
\usepackage{gensymb}
\usepackage{multirow}
\usepackage{color}
\usepackage{wrapfig}
\usepackage[font=small,labelfont=bf]{caption}
%\usepackage{epsfig}
\author{N. Belliveau, G. Chure, J. Theriot, R. Phillips}

\begin{document}
\title{Supplemental Information}
\maketitle

\section{Summary of Proteome Datasets.}

Here we briefly summarize the datasets that were used for the work of the
main text. The goal of this section is to give an overview of each
dataset considered, including the main experimental details, and to provide a
more detailed look at how well each compares.

Table \ref{table:datasets} provides an overview of the proteomic datasets that
we foudn in the literatrue. These are predominately mass spectrometry based,
with the exception of the work from Li et al. (2014) which used ribosomal
profiling, and the fluorescence-based counting done in Taniguchi et al. (2010).
With the exception of Schmidt et al. and Taniguchi et al., the general strategy
taken was to quantify fractional abundance of each protein and then to convert
these to absolute abundance by multiplying this fraction by the bulk measusured
total cellular protein abundance. Note that the work of Peebo et al. (2014) did
not perform any measurement of cell count or volume, and thus were only able to
report cellular protein concentration. A key distinction in the work of Schmidt
et al. (2016) is that 41 enzymes covering over four orders of magnitude in
cellular abundance were used for absolute quantification. Specifically,
synthetic peptides were generated for each of the 41 enzymes and used to provide
a calibration between measured mass spectrometry intensities and absolute
protein abundances (using stable isotope dilution (SID) and selected reaction
monitoring (SRM), though the details of this are beyond the scope of this
section).


\begin{center}
\begin{tabular}{ || c | c | c || }
\hline
Author & Method & Reported Quantity & fractional coverage (by count) & fractional coverage (by mass) \\
\hline\hline
Taniguchi et al. (2010) & YFP-fusion, cell fluorescence & fg/copies per cell & & \\
\hline
Valgepea et al. (2012) & Mass spectrometry & fg/copies per cell & & \\
\hline
Peebo et al. (2014) & Mass spectrometry & fg/copies per fL & & \\
\hline
Li et al. (2014) & Ribosomal profiling & protein synthesis rate & & \\
\hline
Soufi et al. (2015) & Mass spectrometry & fg/copies per cell & &\\
\hline
Schmidt et al. (2016) & Mass spectrometry & fg/copies per cell & & \\
\hline
Caglar et al. (2017) & Mass spectrometry & relative abundance & &\\
\hline
\end{tabular}
\label{table:datasets}
\end{center}

Figure \ref{} shows the distribution in reported protein abundance - 


\section{Adjustments to Copy Number Data.}

%
% \begin{figure}[H]
% 		\centering
%   \begin{tabular}{ l  l }
% 	  A) & B) \\
% 	      \includegraphics[width=0.55\textwidth]{SILAC_corr_O3_2.pdf} &
% 	       \includegraphics[width=0.55\textwidth]{deltaSILAC_corr_O3.pdf} \\
%   \end{tabular}
%   \caption{}
%   \label{fig:HQ1Q2}
% \end{figure}


\end{document}
