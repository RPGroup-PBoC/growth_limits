% \subsection{RNA Synthesis}\label{sec:RNA_synthesis}
We now turn our attention to the transcription of DNA to form RNA. Here we focus
on the synthesis of rRNA, which make up the majority of RNA in the cell, and
discuss the synthesis of mRNA and tRNA further in the Supplemental Section
"Additional Process of the Central Dogma". rRNA serves as the catalytic and
structural component of the ribosome, comprising approximately 2/3 of the total
ribosomal mass, and is decorated with $\approx$ 50 ribosomal proteins. Each
ribosome contains three rRNA molecules of lengths 120, 1542, and 2904
nucleotides (BNID: 108093), meaning each ribosome contains $\approx$ 4500
nucleotides overall. \textit{In vivo} measurements of the kinetics of rRNA
transcription have revealed that RNA polymerases are loaded onto the promoter of
an rRNA gene at a rate of $\approx$ 1 per second (BNID: 111997, 102362). If RNA
polymerases are constantly loaded at this rate, then we can assume that
$\approx$ 1 functional rRNA unit is synthesized per second per rRNA operon.  At
a growth rate of $\approx$ 0.5 hr$^{-1}$, the average cell has $\approx$ 1 copy
of its chromosome and therefore approximately $\approx$ 7 copies of the rRNA
operons, producing $\approx$ 7 rRNA units per second. With a 5000 second
division time, this means the cell is able to generate around 3 $\times$ 10$^4$
functional rRNA units, comparable within an order of magnitude to the number of
ribosomes per cell.

How many RNA polymerases are then needed to constantly transcribe the required
rRNA? If one polymerase is loaded once every second on average (BNID: 111997),
and the transcription rate is $\approx$ 40 nucleotides per second (BNID:
101094), then the typical spacing between polymerases will be $\approx$ 40
nucleotides. With a total length of $\approx$ 4500 nucleotides per operon and 7
operons per cell, the number of RNA polymerases transcribing rRNA at any given
time is then $\approx$ 1000 per cell. We also find that cells require on the
order of another $\approx$ 400 RNAP for the  synthesis of mRNA and tRNA,
predicting a total of  $\approx$ 1500 RNAP to satisfy its transcriptional
demands. As is revealed in \FIG{central_dogma}(C), this estimate is about an
order of magnitude below the observed number of RNA polymerase complexes per
cell ($\approx$ 5000 - 7000). Consistent with this, a majority of RNAP is known
to be  non-specifically bound to DNA during its search for promoters from which
to begin transcription \citep{klumpp2008, patrick2015},  In
\FIG{central_dogma}(D), we find that the predicted RNA polymerase copy number
indeed is more comparable with the abundance of $\sigma$-70 (RpoD), the primary
workhorse sigma factor for transcription in \textit{E. coli}. We can conclude
that the observed RNA polymerase abundances are generally sufficient for what
appears needed for growth.

% The difference between the estimated number of RNA
% polymerase needed for transcription and these observations, however, are
% consistent with literature revealing that $\approx$ 80\% of RNA polymerases in
% \textit{E. coli} are not transcriptionally active \citep{patrick2015}.

% Our estimates also neglect other mechanistic features of transcription and
% transcriptional initiation more broadly. For example, we acknowledge that a
% major fraction of the RNAP pool is non-specifically bound to DNA during its
% search for promoters from which to begin transcription \citep{klumpp2008}.
% Furthermore, we ignore the obstacles that RNA polymerase and DNA polymerase
% present to each other as they move along the DNA \citep{finkelstein2013}.
% Additionally, while they represent the core machinery for transcription, RNA
% polymerase is not sufficient to initiate transcription. Initiation of
% transcription is often dependent on the presence of $\sigma$-factors, protein
% cofactors that bind directly to the polymerase \citep{browning2016} and aid in
% promoter recognition. In \FIGSUPP[RNA_synthesis]{sigma_70}, we show that the
% predicted RNA polymerase copy number indeed is more comparable with the
% abundance of $\sigma$-70 (RpoD), the primary sigma factor in \textit{E. coli}.
% There therefore remains more to be investigated as to what sets the observed
% abundance of RNA polymerase in these proteomic data sets. However, we conclude
% that the observed RNA polymerase abundances are generally in excess of what
% appears to be needed for growth, suggesting that the synthesis of RNA polymerase
% themselves are not particularly limiting.
%
