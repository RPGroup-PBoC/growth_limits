\documentclass[11pt]{article}
\setlength{\topmargin}{-0.25in}
\setlength{\textheight}{8.75in}
\setlength{\oddsidemargin}{.125in}
\setlength{\textwidth}{6.25in}
\linespread{1.4}
\usepackage{graphicx}
\usepackage{chemmacros}
\usepackage{amsmath}
\usepackage{float}
\usepackage{gensymb}
\usepackage{multirow}
\usepackage{color}
\usepackage{wrapfig}
\usepackage[font=small,labelfont=bf]{caption}
%\usepackage{epsfig}
\author{N. Belliveau, G. Chure, J. Theriot, R. Phillips}

\begin{document}
\title{Supplemental Information}
\maketitle

\section{Summary of Proteome Datasets.}

Here we briefly summarize the datasets that were used for the work of the
main text. The goal of this section is to give an overview of each
dataset considered, including the main experimental details, and to provide a
more detailed look at how well each compares.

Table \ref{table:datasets} provides an overview of the proteomic datasets that
we foudn in the literatrue. These are predominately mass spectrometry based,
with the exception of the work from Li et al. (2014) which used ribosomal
profiling, and the fluorescence-based counting done in Taniguchi et al. (2010).
With the exception of Schmidt et al. and Taniguchi et al., the general strategy
taken was to quantify fractional abundance of each protein and then to convert
these to absolute abundance by multiplying this fraction by the bulk measusured
total cellular protein abundance. Note that the work of Peebo et al. (2014) did
not perform any measurement of cell count or volume, and thus were only able to
report cellular protein concentration. A key distinction in the work of Schmidt
et al. (2016) is that 41 enzymes covering over four orders of magnitude in
cellular abundance were used for absolute quantification. Specifically,
synthetic peptides were generated for each of the 41 enzymes and used to provide
a calibration between measured mass spectrometry intensities and absolute
protein abundances (using stable isotope dilution (SID) and selected reaction
monitoring (SRM), though the details of this are beyond the scope of this
section).


\begin{center}
\begin{tabular}{ || c | c | c || }
\hline
Author & Method & strain & Reported Quantity & fractional coverage (by count) & fractional coverage (by mass) \\
\hline\hline
Taniguchi et al. (2010) & & YFP-fusion, cell fluorescence & fg/copies per cell & & \\
\hline
Valgepea et al. (2012) & & Mass spectrometry & fg/copies per cell & & \\
\hline
Peebo et al. (2014) & & Mass spectrometry & fg/copies per fL & & \\
\hline
Li et al. (2014) & & Ribosomal profiling & protein synthesis rate & & \\
\hline
Soufi et al. (2015) & & Mass spectrometry & fg/copies per cell & &\\
\hline
Schmidt et al. (2016) & & Mass spectrometry & fg/copies per cell & & \\
\hline
Caglar et al. (2017) & & Mass spectrometry & relative abundance & &\\
\hline
\end{tabular}
\label{table:datasets}
\end{center}

Figure \ref{} shows the distribution in reported protein abundance for   a  subset
of  the data.

Figure \ref{} compares each dataset to the copy numbers from Schmidt et al.,
grown in M9 minimal media supplemented with glucose.


\section{Adjustments to Copy Number Data.}

% NB: need to make summary figure with 2x2 panels; top row A) reported total protein per cells
% B) protein concentration. bottom row, C) new total protein mass per cell,
% C) new protein concentration.

% NB: It may be helpful to appeal to the 'classic growth laws' early on in this text
% as a rational to guide thinking with expections about total cell mass, cell volume w.r.t.
% growth rate.

The datasets were generated using varied strategies that include a range of
bacterial growth conditions,  different {\it e. coli} strains, and for those
that report quantities on a cell basis, different strategies to count cells for
normalization. It was therefore important to consider if any obvious
discrepencies exist in the data and whether these might be reasonably dealt with
to make the compilated data internally consistent. - example of yeast proteome
data corrections. However, given the work of [cite] and others, there are prior
expectations including an increase in total protein mass per cell with
increasing growth rate. We were therefore hesitent to apply any global
renormalizations, or scaling that does not take into account such trends. Figure
\ref{} shows the total protein  mass reported as a function of growth rate for
each experiment. Indeed, with the exception of the work of Peebo et al., the
total mass per cell is generally consistent as a fuction of growth rate. One
parameter that we do not expect to change substantially across growth conditions
is cellular protein concentration; any discrepencies may reflect differences in
protein extraction efficency.

In the remainder of this section we describe how we correct for differences in
cellular protein concentration within each inidividual dataset. This  strategy
was already applied in the work of Schmidt et al., in particularr due to
concerns  over lower protein extraction efficiency in growth conditions like
stationary phase.  However, another complicating factor that became apparent is
a descrpency in expected versus reported cell volume that required additional
care and is further described below. Lastly, the  data from Peebo {\it et al.}
required additional care due to a lack quantities on a cell basis, and we
consider this work seperately.


\subsection{Corrections to Enforce a Consistent Cellular Protein Concentration}

% NB: It may be useful to note that none of this should have any effect on the relative
% abundances found in each dataset.


In order to correct for potential differences in protein extraction across
growth conditions, Schmidt {\it et al.} required that all cellular concentrations
be identical. This expectation is justified by work from [ref??]. We can also rationalize
this from the knowledge that the fraction  of protein dry mass in a cell does not
vary substantially [ref?].

NB: need to rework - I want to show that protein concentration shouldn't vary very much (and suggest some
bounds if possible).
With this in mind, consider estimating the cellular protein
concentration in a cell. This will be given by the (total mass of a cell) *
(fraction of dry mass) * (fraction  of dry mass that is protein) / (cell
volume). The total mass of a cell can also be written as (cell volume) * (cell
density) . And so we are  left with (cell density) * (fraction of dry mass) *
(fraction  of dry mass that is protein) . The density is roughly 1.1 g/mL, which
is relatively fixed, the dry  mass is about 30\%,  while about 55\% of this  is
protein.


From the work of Schmidt et al. they report an ability to consistently get high
protein yield from cells grown in M9 minimal media supplemented with glucose,
whereas other growth  conditions such as stationary  phase growth being more
troublesome. Therefore in  order to account any protein loss during extraction,
they use growth in glucose as a
constant reference for which total protein concentration in all other growth
conditions should match. This is shown in Figure \ref{}A. One challenge in
performing this calculation is that cell volume must be known; the authors use
volumes that were  measured by flow cytometry in previous work [cite]. These
volumes are shown in Figure \ref{}B. While it is difficult to assess the
accuracy of these numbers, we find them to be quite inconsistent with the
expected scaling that is reported by Taheri-Araghi et al. (2015) [and other work?].

Since cell volume is not determined in all studies, and to be  consistent
throughout, we instead use the predicted cell volumes that were carefully
measured as a function of growth rate in the work of Taheri-Araghi et al.
(2015). Dealing with each author's  proteomic data seperately, we follow the
approach employed by Schmidt et al., and require that the cellular protein
concentration be consistent across each set of proteomic data. This  amounts to
calculating the total protein mass per cell, divided by the estimated cell
volume,  and then applying a correction factor to correct for any descrepency in
protein concentration.  For each of the author's work, we have selected a
reference growth condition with growth rate of about $\lambda \approx 0.5
hr^{-1}$, which is the growth rate for growth in M9 minimal media supplemented
with glucose.


\subsection{Peebo {\it et al.}: Conversion from copies/ fL to copies per cell}

In the work of Peebo {\it et al.}, the authors only report protein concentration.
In  order to estimate protein per cell, we multiple  these concentrations by
expected cell volumes  using the predictions from  Taheri-Araghi et al. This is
shown in Figure \ref{}A, where we see that reported mass is substantially lower than
the other work considered here; as well as work from others [Sinauer, 1990].

Indeed, both Schmidt et al. and Li et al. reported a total protein mass of about
250 fg per cell at a growth rate of about $\lambda \approx 0.5 hr^{-1}$ ( M9
minimal media with glucose and MOPS minimal media, respectively). Given this
descrepency, in addition to requiring that cellular protein concentration be
internally consistent across the growth conditions they reported on, we also
required that total cellular mass be consistent with the work Schmidt et al. and
Li et al. This amounted to performing a linear regression between total protein
mass and growth rate, and using this to scale the Peebo et al. dataset according
to this trend. 


 % cell biology by the numebrs: Overall macromolecular composition of an average
 % E. coli cell in aerobic balanced growth at 37°C in glucose minimal medium,
 % with doubling time of 40 minutes and 1 pg cell wet weight (≈0.9 μm^3 cell
 % volume). Adapted with modifications from F. C. Neidhardt et al., “Physiology
 % of the bacterial cell”, Sinauer, 1990 (BNID 104954). Modifications included
 % increasing cell dry weight from 284 fg to 300 fg and total cell mass from 950
 % to 1000 fg as

%
% \begin{figure}[H]
% 		\centering
%   \begin{tabular}{ l  l }
% 	  A) & B) \\
% 	      \includegraphics[width=0.55\textwidth]{SILAC_corr_O3_2.pdf} &
% 	       \includegraphics[width=0.55\textwidth]{deltaSILAC_corr_O3.pdf} \\
%   \end{tabular}
%   \caption{}
%   \label{fig:HQ1Q2}
% \end{figure}


\end{document}
