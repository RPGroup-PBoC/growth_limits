\section{Discussion}
Continued experimental and technological improvements have led to a treasure
trove of quantitative biological data \citep{hui2015, schmidt2016, si2017,
gallagher2020, peebo2015, valgepea2013}, and an ever advancing molecular view
and mechanistic understanding of the constituents that support bacterial growth
\citep{taheriaraghi2015, morgenstein2015, si2019, karr2012, kostinski2020,
macklin2020}. In this work we have compiled what we believe to be the
state-of-the-art knowledge on proteomic copy number across a broad range of
growth conditions in \textit{E. coli}. We have made this data accessible through
a \href{https://github.com/RPGroup-PBoC/growth_limits}{GitHub repository}, and
an \href{https://rpgroup.caltech.edu/growth_limits//data_explorer}{interactive
figure} that allows exploration of specific protein and protein complex copy
numbers.

Through a series of order-of-magnitude estimates that traverse key
steps in the bacterial cell cycle, this proteomic data has been a resource to
guide our understanding of two key questions: what biological processes limit
the absolute speed limit of bacterial growth, and how do cells alter their
molecular constituents as a function of changes in growth rate or nutrient
availability? While not exhaustive, our series of estimates provide insight on
the scales of macromolecular complex abundance across four classes of cellular
processes -- the transport of nutrients, the production of energy, the synthesis
of the membrane and cell wall, and the numerous steps of the central dogma.

In general, the copy numbers of the complexes involved in these processes were
in reasonable agreement with our order-of-magnitude estimates. Since many of these
estimates represent soft lower-bound quantities, this suggests that cells do not
express proteins grossly in excess of what is needed for a particular growth
rate. Several exceptions, however, also highlight the dichotomy between a
proteome that appears to "optimize" expression according to growth rate and one
that must be able to quickly adapt to environments of different nutritional
quality. Take, for example, the expression of carbon transporters. Shown in
\FIG{carbon_tport}(B), we find that cells always express a similar number of
glucose transporters irrespective of growth condition. At the same time, it is
interesting to note that many of the alternative carbon transporters are still
expressed in low but non-zero numbers ($\approx$ 10-100 copies per cell) across
growth conditions. This may relate to the regulatory configuration for many of
these operons, which require the presence of a metabolite signal in order for
alternative carbon utilization operons to be induced \citep{monod1949,
laxhuber2020}. Furthermore, upon induction, these transporters are expressed and
present in abundances in close agreement with a simple estimate.

Of the processes illustrated in \FIG{categories}, we arrive at a
ribosome-centric view of cellular growth rate control. This is in some sense
unsurprising given the long-held observation that \textit{E. coli} and many
other organisms vary their ribosomal abundance as a function of growth
conditions and growth rate \citep{scott2010, metzlraz2017}. However, through our
dialogue with the proteomic data, two additional key points emerge. The first
relates to our question of what process sets the absolute speed limit of
bacterial growth. While a cell can parallelize many of its processes simply by
increasing the abundance of specific proteins or firing multiple rounds of DNA
replication, this is not so for synthesis of ribosomes
[\FIG{ribosome_limit}(A)]. The translation time for each ribosome [$\approx$ 7
min, \cite{dill2011}] places an inherent limit on the growth rate that can only
be surpassed if the cell were to increase their polypeptide elongation rate, or
if they could reduce the total protein and rRNA mass of the ribosome. The second
point relates to the long-observed correlations between growth rate and cell
size \citep{schaechter1958, si2017}, and between growth rate and ribosomal mass
fraction. While both trends have sparked tremendous curiosity and driven
substantial amounts of research in their own regards, these relationships are
themselves intertwined. In particular, it is the need for cells to increase
their absolute number of ribosomes under conditions of rapid growth that require
cells to also grow in size. Further experiments are needed to test the validity
of this hypothesis. In particular, we believe that the change in growth rate in
response to translation-inhibitory drugs (such as chloramphenicol) could be
quantitatively predicted, given one had precision measurement of the relevant
parameters, including the fraction of actively translating ribosomes $f_a$ and
changes in the metabolic capacity of the cell (i.e. the rate that amino acids can be made
available) for a particular growth condition.

While the generation of new ribosomes plays a dominant role in growth rate
control, there exist other physical limits to the function of cellular
processes. One of the key motivations for considering energy production was the
physical constraints on total volume and surface area as cells vary their size
\citep{harris2018, ojkic2019}. As \textit{E. coli} get larger at faster growth
rates, an additional constraint begins to arise in energy production and
nutrient uptake due to the relative decrease in total surface area, where ATP is
predominantly produced \citep{szenk2017}. Specifically, the cell interior
requires an amount of energy that scales cubically with cell size, but the
available surface area only grows quadratically [\FIG{energy_scaling}(A)]. While
this threshold does not appear to be met for \textit{E. coli} cells growing at 2
hr$^{-1}$ or less, it highlights an additional constraint on growth given the
apparent need to increase cell size in order to grow faster. This limit is
relevant even to eukaryotic organisms, whose mitochondria exhibit convoluted
membrane structures that nevertheless remain bacteria-sized organelles
\citep{guo2018}. In the context of bacterial growth and energy production more
generally, we have mainly limited our analysis to the aerobic growth conditions
associated with the proteomic data and further consideration will be needed for
anaerobic growth.1

This work is by no means meant to be a complete dissection of bacterial
growth rate control, and there are many aspects of the bacterial proteome and
growth that we neglected to consider. For example, other recent work
\citep{liebermeister2014, hui2015, schmidt2016} has explored how the proteome is
structured and how that structure depends on growth rate. In the work of
\cite{hui2015}, the authors coarse-grained the proteome into six discrete
categories being related to either translation, catabolism, anabolism, and
others related to signaling and core metabolism. The relative mass fraction of
the proteome occupied by each sector could be modulated by external application
of drugs or simply by changing the nutritional content of the medium. While we
have explored how the quantities of individual complexes are related to cell
growth, we acknowledge that higher-order interactions between groups of
complexes or metabolic networks at a systems-level may reveal additional
insights into how these growth-rate dependences  are mechanistically achieved.
Furthermore, while we anticipate the conclusions summarized here are applicable
to a wide collection of bacteria with similar lifestyles as \textit{E. coli},
other bacteria and archaea may have evolved other strategies that were not
considered. Further experiments with the level of rigor now possible in
\textit{E. coli} will need to be performed in a variety of microbial organisms
to learn more about how regulation of proteomic composition and  growth rate
control has evolved over the past 3.5 billion years.
