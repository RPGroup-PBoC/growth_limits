Recent years have seen an experimental deluge interrogating the relationship
between bacterial growth rate, cell size, and protein content. However, we still
lack a rigorous understanding of what sets the scale of these quantities and
when protein abundances should depend on growth rate. Here, we seek to
quantitatively understand this relationship across a collection of
\textit{Escherichia coli} proteomic data covering $\approx$ 4000 proteins and 36
growth rates. We estimate the basic requirements and physical constraints on
steady-state growth by considering key processes in cellular physiology. In
contrast to perspectives that specific processes limit growth rate or dictate
cell size, our analysis suggests that cells are predominantly tuned for the task
of cell doubling across a continuum of growth rates. Importantly, a theoretical
inability to parallelize ribosomal synthesis places a firm limit on the
achievable growth rate that is reflected under moderate to fast growth rates. We
expand on this assessment by exploring a model of proteomic regulation as a
function of nutrient supply, revealing a mechanism that ties size and growth
rate to proteomic content.
