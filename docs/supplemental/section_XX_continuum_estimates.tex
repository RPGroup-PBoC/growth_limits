\section{Extending Estimates to a Continuum of Growth Rates}
In the main text, we considered a standard stopwatch of 5000 s to estimate the
abundance of the various protein complexes considered. In addition to point
estimates, we also showed the estimate as a function of growth rate as
transparent grey curves. In this section, we elaborate on this continuum
estimate and compare and contrast the approach to the point estimate procedure. 

\subsection{Estimation of the total cell mass}
For many of the processes estimated in the main text we relied on a cellular dry
mass of $\approx 300$ fg from which we computed elemental and protein fractions
using knowledge of fractional composition of the dry mass. At modest growth
rates, such as the 5000 s doubling time used in the main text, this is a
reasonable number to use as the typical cell mass is $\approx$ 1 pg and
\textit{E. coli} cells can approximated as 70\% water by volume. However, as we
have shown in this supplemental information, the cell size and therefore cell
volume is highly dependent on the growth rate. This means that a dry mass of 300
fg cannot be used reliably across all growth rates. 


Rather, using 



