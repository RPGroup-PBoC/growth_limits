Recent years have seen a deluge of experiments dissecting the relationship
between bacterial growth rate, cell size, and protein content, quantifying
the abundances of single proteins across growth conditions with unprecedented
resolution. However, we still lack a rigorous understanding of what sets the
scale of these measurements why single protein abundances do (or do not)
depend on growth rate. Here, we seek to quantitatively understand the scales
of the observations in a collection of \textit{Escherichia coli} proteomic
data sets covering $\approx$ 4000 proteins and 31 growth conditions. We
estimate the abundances of complexes needed for nutrient transport, energy
generation, cell envelope biogenesis, and the processes of the central dogma,
from which ribosome biogenesis emerges as a primary determinant of growth
rate. We conclude by exploring a model of ribosomal regulation as a function
of the nutrient supply, revealing a mechanism tying cell size and growth rate
to ribosomal content.
