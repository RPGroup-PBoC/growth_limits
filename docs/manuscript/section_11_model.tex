While this dependence between cell size and ribosomal abundance is apparent
across moderate to fast growth rates, it is worth noting that this scaling is
likely to change at slow growth rates (below $\lambda \approx
0.5\,\text{hr}^{-1}$). Here, the number of ribosomes $R$ no longer reflects the
cell's protein synthesis capacity, so far taken to be $r_t
\times R$. Instead, cells reduce the number of actively translating
ribosomes through the additional regulatory control of the small-molecule
alarmone, guanosine pentaphosphate [(p)ppGpp] \citep{dai2016} [more citations].

% This overabundance of ribosomes
% provides different challenges on the ability of the cell to maintain steady-state
% growth under limiting nutrient conditions, and in Supplemental
% Section XX we consider this slow growth regime further.

\subsection{\textit{E. coli} Maximizes Steady-State Growth
Rate by Tuning both Ribosomal Content and Translation Activity.}

The translation-limited growth rate (\EQ{translation_limit_growth_rate})
highlights the necessity for a cell to increase its ribosomal fraction to
maximize its growth rate. However, it ignores the absolute constraints of a
growing cell, which in particular, require resources to support each translating
ribosomes and the space to house them. For almost every one of our estimates,
the proteomic data suggest that cells predominantly vary their protein
abundances in order to keep pace with the demands of a growing cell. In this
final section we consider a minimal model of growth rate control. We use it to
provide intution into the additional constraints that arise when growth rate and
ribosomal abudance is intimately tied to cell size.

[do I talk about ppGpp and nutrient sensing? Maybe later once I show the trends]

In order for cells to maxmize their rate of protein synthesis, $r_t
\times R$, ribosomes must be able to rapidly match codons with their correct
amino-acyl tRNAs. This requires synthesis of (or import, for rich media) amino acids that
maintain the pool of amino-acyl tRNAs. To proceed, we therefore consider
this maintenance as simply a balance between the the supply of amino acids
by metabolic proteins and transporters, at a rate of $r_{aa}$ in units of amino acids per second,
and consumption by ribosomes at a rate of $r_t
\times R \times f_a$. The addition factor of $f_a$ accounts for the possibility
that





[important point is that as resources become limiting, cells are able to tune and minimize
the entire cell mass - which enables them to grow faster]

[Discuss implications of findings so far. All other components being tuned (mostly) in the required proportions; and also that the achievable growth rate is ultimately set by ribosomes. ]

[The apparent constraint that cells MUST get larger in order to grow faster places a particular constrain on why a cell would vary its ribosomal fraction in the first place.]

[Also present the evidence from literature that there is a prominent role for aa/nutrient sensing that appears to mediate X, Y, and Z. ]

[Then go on to propose the way to understand what’s going on with our model.
    - I think maybe it’s easiest to understand it is we start by considering the balance that must be maintained between metabolic activity and ribosomal activity.
    - Then write down a model that relates how elongation rate will relate to this.]



gives no intuition into the additional constraints of a growing
cell that we've consider throughout our estimates.



quantifies the maximum rate of growth but ignores the additional constraint set  that there is a maximum rate us with a maximum

[Figure idea: PART (A) Maybe we can consider that r_t*R sets the maximum rate of growth; but you also need to feed that engine and build the house to fit them.]

[ When I get into slow growth regime; consider that IF R scales with <#ori>, ribosomes will have increasingly longer wait times for protein synthesis. Which is bad.
    - By also decreasing the fraction of actively translating ribosomes, cells can grow faster is also super important. ]




As a final comment, it has recently been shown that growth in a (p)ppGpp null
strain also lacked both the condition-dependent changes in $\langle$\#
ori$\rangle$ as well as changes in cell size across different growth condition.
Instead, cells always exhibited a high ratio of $\langle$\# ori$\rangle$ to
$\langle$\# ter$\rangle$, irrespective of growth rate, and a cell size that was
more consistent with a fast growth state where (p)ppGpp levels are normally low
\citep{fernandezcoll2020} and ribosomal fraction is high \citep{zhu2019}. There
is also evidence that this may be achived through inhibition of DNA replication
initiation \citep{kraemer2019}. These observations raise the possibility that
(p)ppGpp may be playing a causal role in tuning $\langle$\# ori$\rangle$ and
cell size, which ultimately allows the cell to vary its ribosomal content
according to nutrient availability.
