\documentclass[11pt]{article}
\setlength{\topmargin}{-0.25in}
\setlength{\textheight}{8.75in}
\setlength{\oddsidemargin}{.125in}
\setlength{\textwidth}{6.25in}
\linespread{1.4}
\usepackage{graphicx}
\usepackage{chemmacros}
\usepackage{amsmath}
\usepackage{float}
\usepackage{gensymb}
\usepackage{multirow}
\usepackage{color}
\usepackage{wrapfig}
\usepackage[font=small,labelfont=bf]{caption}
%\usepackage{epsfig}
\author{N. Belliveau, G. Chure, J. Theriot, R. Phillips}

\begin{document}
\title{Supplemental Information}
\maketitle

\section{Summary of Proteome Datasets.}

Here we briefly summarize the datasets that were used for the work of the
main text. The goal of this section is to give an overview of each
dataset considered, including the main experimental details, and to provide a
more detailed look at how well each compares.

Table \ref{table:datasets} provides an overview of the proteomic datasets that
we foudn in the literatrue. These are predominately mass spectrometry based,
with the exception of the work from Li et al. (2014) which used ribosomal
profiling, and the fluorescence-based counting done in Taniguchi et al. (2010).
With the exception of Schmidt et al. and Taniguchi et al., the general strategy
taken was to quantify fractional abundance of each protein and then to convert
these to absolute abundance by multiplying this fraction by the bulk measusured
total cellular protein abundance. Note that the work of Peebo et al. (2014) did
not perform any measurement of cell count or volume, and thus were only able to
report cellular protein concentration. A key distinction in the work of Schmidt
et al. (2016) is that 41 enzymes covering over four orders of magnitude in
cellular abundance were used for absolute quantification. Specifically,
synthetic peptides were generated for each of the 41 enzymes and used to provide
a calibration between measured mass spectrometry intensities and absolute
protein abundances (using stable isotope dilution (SID) and selected reaction
monitoring (SRM), though the details of this are beyond the scope of this
section).


\begin{center}
\begin{tabular}{ || c | c | c || }
\hline
Author & Method & strain & Reported Quantity & fractional coverage (by count) & fractional coverage (by mass) \\
\hline\hline
Taniguchi et al. (2010) & & YFP-fusion, cell fluorescence & fg/copies per cell & & \\
\hline
Valgepea et al. (2012) & & Mass spectrometry & fg/copies per cell & & \\
\hline
Peebo et al. (2014) & & Mass spectrometry & fg/copies per fL & & \\
\hline
Li et al. (2014) & & Ribosomal profiling & protein synthesis rate & & \\
\hline
Soufi et al. (2015) & & Mass spectrometry & fg/copies per cell & &\\
\hline
Schmidt et al. (2016) & & Mass spectrometry & fg/copies per cell & & \\
\hline
Caglar et al. (2017) & & Mass spectrometry & relative abundance & &\\
\hline
\end{tabular}
\label{table:datasets}
\end{center}

Figure \ref{} shows the distribution in reported protein abundance for   a  subset
of  the data.

Figure \ref{} compares each dataset to the copy numbers from Schmidt et al.,
grown in M9 minimal media supplemented with glucose.


\section{Adjustments to Copy Number Data.}

The datasets were generated using varied strategies that include a range of
bacterial growth conditions,  different {\it e. coli} strains, and for those
that report quantities on a cell basis, different strategies to count cells for
normalization. It was therefore important to consider if any obvious
discrepencies exist in the data and whether these might be reasonably dealt with
to make the compilated data internally consistent. - example of yeast proteome
data corrections. However, given the work of [cite] and others, there are prior
expectations including an increase in total protein mass per cell with
increasing growth rate. We were therefore hesitent to apply any global
renormalizations, or scaling that does not take into account such trends. Figure
\ref{} shows the total protein  mass reported as a function of growth rate for
each experiment. Indeed, with the exception of the work of Peebo et al., the
total mass per cell is generally consistent as a fuction of growth rate. One
parameter that we do not expect to change substantially across growth conditions
is cellular protein concentration; any discrepencies may reflect differences in
protein extraction efficency.

In the remainder of this section we describe how we correct for differences in
cellular protein concentration within each inidividual dataset. This  strategy
was already applied in the work of Schmidt et al., in particularr due to
concerns  over lower protein extraction efficiency in growth conditions like
stationary phase.  However, another complicating factor that became apparent is
a descrpency in expected versus reported cell volume that required additional
care and is further described below. Lastly, the  data from Peebo {\it et al.}
required additional care due to a lack quantities on a cell basis, and we
consider this work seperately.


\subsection{Corrections to Enforce a Consistent Cellular Protein Concentration}

In order to correction for potential differences in protein extraction across
growth conditions, Schmidt {\it et al.} required that all cellular concentrations
be identical. In their  work they note that under growth in -

This also appears to be consistent with the work in Li {\it et al.}



\subsection{Peebo {\it et al.}: Conversion from copies/ fL to copies per cell}


%
% \begin{figure}[H]
% 		\centering
%   \begin{tabular}{ l  l }
% 	  A) & B) \\
% 	      \includegraphics[width=0.55\textwidth]{SILAC_corr_O3_2.pdf} &
% 	       \includegraphics[width=0.55\textwidth]{deltaSILAC_corr_O3.pdf} \\
%   \end{tabular}
%   \caption{}
%   \label{fig:HQ1Q2}
% \end{figure}


\end{document}
