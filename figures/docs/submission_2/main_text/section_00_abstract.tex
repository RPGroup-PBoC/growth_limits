Recent years have seen an experimental deluge interrogating the relationship
between bacterial growth rate, cell size, and protein content, quantifying the
abundance of proteins across growth conditions with unprecedented resolution.
However, we still lack a rigorous understanding of what sets the scale of these
quantities and when protein abundances should (or should not) depend on growth
rate. Here, we seek to quantitatively understand this relationship across a
collection of \textit{Escherichia coli} proteomic data covering $\approx$ 4000
proteins and 36 growth rates. We estimate the basic requirements for
steady-state growth by considering key processes in nutrient transport, cell envelope biogenesis, energy
generation, and the central dogma. From these
estimates, ribosome biogenesis emerges as a primary determinant of growth rate.
We expand on this assessment by exploring a model of proteomic regulation as a
function of the nutrient supply, revealing a mechanism that ties cell size and
growth rate to ribosomal content.
