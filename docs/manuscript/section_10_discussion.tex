\section{Discussion}

[Fill in.]

Parallelized DNA replication represents a solution that allows  \textit{E. coli}
to synthesize enough rRNA as it grows faster. That the cell appears to
be synthesizing rRNA near its maximal rate at growth rates above about 0.5 h$^{-1}$
may also have consequences on robust scaling  between
cell size and $\langle$\# ori$\rangle$. As one example, when cells are exposed
to sublethal doses of ribosome-inhibiting drugs like chloramphenicol, there is a
notable increase in the cell's ribosomal fraction (grey points,
\FIG{translation_ecoli_partA}(D)), but otherwise the cell is able to maintain
its cell size according to $\langle$\# ori$\rangle$ \citep{si2017}. While the
presence of chloramphenicol will inhibit protein synthesis, it will also allow
for relatively higher rRNA synthesis due to the longer doubling time. If the
cell can scale it ribosomal protein abundance through feedback from rRNA
synthesis, than we would expect the relative abundance of ribosomes to increase
according to the increase in rRNA synthesis. This type of feedback was
considered long ago \citep{nomura1984} and we consider this possibility further
in Supplemental Appendix XX.


% While it is difficult to distinguish between causality and correlation, the data
% also helps us resolve why cell size, should exhibit an exponential increase with growth
% rate once cells begin to parallelize DNA replication. Specifically,
%
% is consistent with the need for cells to increase their effective rRNA gene
% dosage in order to grow according to the constraint set by Equation 2. Importantly, it
% may also shed some light on the notable increase in ribosomal content
% that is observed when sublethal doses of antibiotics \citep{scott2010, dai2016}.
% Specifically, if rRNA synthesis is rate limiting, and nutrient conditions
% largely dictate the extent of overlapping DNA replication cycles, than addition
% of antibiotic will lengthen the doubling time and allow increased rRNA
% synthesis relative to the rate of cell division. In Supplemental Section XX, we
% consider this further using additional data from \cite{si2017}.


A number of recent papers further highlight the possibility that regulation
(p)ppGpp may be the critical component of the apparent scaling laws in
\textit{E. coli}. In the context of ribosomal activity, increased levels of
(p)ppGpp are associated with lower ribsomal content, and at slow growth
are associated with reduced activity by ribosomes and RNA polymerse \citep{dai2016,
dai2018} [citations for RNAP]. Titration of the cellular (p)ppGpp concetrations (up or down) can
invoke similar proteomic changes reminiscent of those observed under nutrient
limitation \citep{zhu2019}. In light of the limiting dependence of ribosome copy
number on chromosomal gene dosage, it was recently shown that growth in a
(p)ppGpp null strain abolishes both the scaling in cell size  and the
$\langle$\# ori$\rangle$ / $\langle$\# ter$\rangle$ ratio. Instead, cells
exhibited a high $\langle$\# ori$\rangle$ / $\langle$\# ter$\rangle$ closer to 4
and cell size more consistent with a fast growth state where (p)ppGpp levels are
low \citep{fernandezcoll2020}. From these results, perhaps the perspective to
take is that the scaling laws  reflect an attempt for the cell to mitigate its
biological activity according  to available nutrients.
