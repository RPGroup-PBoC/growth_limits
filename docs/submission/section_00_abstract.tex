Recent years have seen a deluge of experiments dissecting the relationship
between bacterial growth rate, cell size, and protein content, quantifying the
abundance of proteins across growth conditions with unprecedented
resolution. However, we still lack a rigorous understanding of what sets the
scale of these quantities and when protein abundances should (or should not)
depend on growth rate. Here, we seek to quantitatively understand this
relationship across a collection of
\textit{Escherichia coli} proteomic data sets covering $\approx$ 4000 proteins
and 31 growth conditions. We estimate the basic requirements for steady-state
growth by considering key processes in nutrient transport, energy generation, cell envelope
biogenesis, and the central dogma, from which ribosome biogenesis
emerges as a primary determinant of growth rate. We conclude by exploring a
model of ribosomal regulation as a function of the nutrient supply, revealing a
mechanism that ties cell size and growth rate to ribosomal content.
