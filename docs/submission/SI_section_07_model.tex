
\section{Derivation of Full Model of Nutrient-limited Growth.}
\label{sec:SI_model}

Here we return to our model of nutrient-limited growth and provide a more complete derivation of elongation rate $r_t$. Several of the steps are reproduced from the main text for completeness, and we begin with our coarse-grained description of peptide elongation, which includes 1) the time required to find and bind each
correct amino-acyl tRNA, and 2) the remaining steps in peptide elongation that
will not depend on the amino acid availability. These time scales are related to the inverse of the elongation rate $r_t$,

\begin{equation}
\frac{1}{r_t} = \frac{1}{k_{on} \alpha [AA]_{\text{eff}}} + \frac{1}{r_{t}^{\text{max}}}.
\end{equation}
where we have assumed that the rate of binding by amino-acyl tRNA $k_{on}$ is
proportional to $[AA]_{\text{eff}}$ by a constant $\alpha$. $r_{t}^{\text{max}}$
refers to the maximum elongation rate. This can be stated more
succinctly in terms of an effective dissociation constant,

\begin{equation}
    K_D = \frac{r_{t}^{\text{max}}}{\alpha k_\text{on}},
\end{equation}
where the elongation rate $r_t$ is now given by

\begin{equation}
r_t = \frac{r_{t}^{\text{max}}}{1 + K_D/[AA]_{\text{eff}}}.
\label{eq:rt_kd_simple}
\end{equation}

As the
rate of amino acid supply, denote by $r_{AA}$, varies with changing nutrient
condtions, the cell maintain [AA]_{\text{eff} by tuning the
rate of amino acid consumption, $r_t \times R \times f_a$).
Under steady-state growth, the amino acid concentration is constant
($\frac{d[AA]_\text{eff}}{dt}=0$). The effective amino acid concentration
$[AA]_{\text{eff}}$ will relate to the rate of amino acid synthesis (or import,
for rich media) and/or tRNA charging, as $r_{AA}$, and the rate of consumption,
$r_t\times R \times f_a$ by,

\begin{equation}
\int_{0}^{t} \frac{d[AA]_{\text{eff}}}{dt} dt =  \int_{0}^{t}([r_{AA}] - [r_t\times R \times f_a]) dt,
\label{eq:aaeff_int}
\end{equation}
where the time from 0 to $t$ is an arbitrary length of time, and the square
brackets indicate concentrations per unit time.
Integrating \EQ{aaeff_int} yields.
\begin{equation}
[AA]_{\text{eff}} =  t([r_{AA}] - [r_t \times R \times f_a]).
\label{eq:aaeff_concs}
\end{equation}

Alternatively, we can state this in terms of absolute ribosome copy number $R$
by considering a unit volume $V$,
\begin{equation}
   [AA]_\text{eff} = \frac{t(r_{AA} - r_t \times R \times f_a)}{V},
   \label{eq:aa_final}
\end{equation}
where $r_{AA}$ is in units of AA per unit time and $r_t$ is in units of AA per
unit time per ribosome. With an expression for $[AA]_\text{eff}$ in hand, we can now solve
\EQ{rt_kd_simple} for $r_t$ which is a quadratic function with a
physically-meaningful root of
\begin{equation}
r_t = \frac{t(r_{AA} + r_t^\text{(max)}Rf_a) + K_D V - \sqrt{(r_{AA}t + r_t^\text{(max)}Rf_at + K_D V)^2 - 4(Rf_at)(r_t^\text{(max)}r_{AA} t)}}{2Rf_at}.
\label{eq:rt_root}
\end{equation}
