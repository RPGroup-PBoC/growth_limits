\section{A Minimal Model of Nutrient-Mediated Growth Rate Control}
While the preceeding subsections highlight a dominant role for ribosomes in
setting the growth rate, our analysis on the whole emphasizes that the total
proteomic content must also change in response to variable growth conditions and
growth rate. In this final section we use a minimal model of growth rate control
to better understand how this interconnection between ribosomal abundance and
total protein influences the observed growth rate.

Here we propose that cells modulate their protein abundance in direct response
to the availability of nutrients in their environment. As noted earlier,
bacteria can modulate ribosomal activity through the secondary-messenger
molecules like (p)ppGpp in poorer nutrient conditions (\FIG{ribosome_limit}(C) -
inset; \cite{dai2016}). Importantly, these secondary-messengers also cause
global changes in transcriptional and translational activity
\citep{hauryliuk2015, zhu2019, Buke2020}. In \textit{E. coli}, amino acid
starvation leads to the accumulation of de-acylated tRNAs at the ribosome's
A-site and a strong increase in (p)ppGpp synthesis activity by the enzyme RelA
\citep{hauryliuk2015}. Along with this,  there is increasing evidence that
(p)ppGpp also acts to inhibit the initiation of DNA replication
\citep{kraemer2019}, providing a potential mechanism to lower $\langle$\#
ori$\rangle$ and maintain a smaller cell size in poorer growth conditions
\citep{fernandezcoll2020}.

To consider this quantitatively, we assume that cells modulate their proteome
($N_\text{pep}$, $R$, $\Phi_R$) to better maximize their rate of peptide
elongation $r_t$. The elongation rate $r_t$ will depend on how quickly the ribosomes can
match codons with their correct amino-acyl tRNA, along with the subsequent steps
of peptide bond formation and translocation. This ultimately depends on the
cellular concentration amino acids, which we treat as a single effective
species, $[AA]_\text{eff}$. In our model, we determine the the rate of peptide
elongation $r_t$ and achievable growth rate as simply depending on the supply of
amino acids (and, therefore, also amino-acyl tRNAs), through a parameter
$r_{AA}$ in units of AA per second, and the rate of amino acid consumption by
protein synthesis ($r_t \times R \times f_a$). This is shown schematically in
\FIG{elongation_rate_model}(A) and derived in Appendix \nameref{SI_model}. Given our observation
that protein synthesis and energy production are not limiting, we
assume that other molecular players required by ribosomes like elongation
factors and GTP are available in sufficient abundance.

In \FIG{elongation_rate_model}(B), we illustrate how the elongation rate will
depend on the ribosomal copy number. Here, we have considered an arbitrarily
chosen $r_{AA} = 5\times 10^6$ AA $\cdot$ s$^{-1} \cdot$ \textmu m$^{-3}$ and
$f_a = 1$ for a unit cell volume $V = 1$fL. At low ribosome copy numbers, the
observed elongation rate is dependent primarily on $[AA]_\text{eff}$ through
$r_{AA}$ [as $r_t^{\text{max}} \times R \times f_a << r_{AA}$, point (1) in
\FIG{elongation_rate_model}(B)]. As the ribosome copy number is increased
such that the amino acid supply rate and consumption rate are nearly equal
[point (2) in \FIG{elongation_rate_model}(B)], the observed elongation rate
begins to decrease sharply. When the ribosome copy number is increased even
further, consumption at the maximum elongation rate exceeds the supply rate,
yielding a significantly reduced elongation rate [point (3) in
\FIG{elongation_rate_model}{B)]. While the elongation rate will always be
dominated by the amino acid supply rate at sufficiently low ribosome copy
numbers, the elongation rate at larger ribosome abundances can be increased
by tuning $f_a$ such that not all ribosomes are elongating, reducing the
total consumption rate.

\begin{figure}
    \centering{
        \includegraphics{main_figs/elongation_model.pdf}
        \caption{\textbf{A minimal model of growth rate control under
        nutrient limitation.} (A) We consider a unit volume of cellular material
        composed of amino acids (colored spheres) provided at a supply rate
        $r_{AA}$. These amino acids are polymerized by a pool of ribosomes
        (brown blobs) at a rate $r_t \times R \times f_a$, where $r_t$ is the
        elongation rate, $R$ is the ribosome copy number in the unit volume, and
        $f_a$ is the fraction of those ribosomes actively translating. (B) The
        observed elongation rate is plotted as a function of ribosomes. The three points correspond to three regimes of
        ribosome copy numbers and are shown schematically on the left-hand side.
        The region of the curve shown as dashed lines represents a non-physical
        copy number, but is shown for illustrative purposes. This curve was
        generated using the parameters $r_{AA} = 5 \times 10^6$ AA / s,
        $r_t^\text{(max)} = 17.1$ AA / s, $f_a = 1$, and a unit cell volume of $V = 1$fL.
        See Appendix \nameref{SI_model} for additional model details. (C) The cellular growth
        rate is plotted as a function of total cellular ribosome copy number for
        different cellular amino acid supply rates, with blue and green curves
        corresponding to low and high supply rates, repsectively. As the
        ribosome copy number is increased, so too is the cell size and total
        protein abundance $N_\text{pep}$. We direct the reader to the Suppemental Information
        for discussion on the inference of the realtionship between cell
        size, number of peptide bonds, and ribosome copy number.}
        \label{fig:elongation_rate_model}
    }
\end{figure}

\subsubsection{Optimal Ribosomal Content and Cell Size Depend on Nutrient
Availability and Metabolic Capacity}
To relate elongation rate to growth rate, we constrain the set of parameters
based on our available proteomic measurements; namely, we restrict the values of
$R$, $N_{pep}$, and cell size to those associated with the amalgamated proteomic
data (described in Appendix \nameref{sec:estimate_protein_per_cell}). We then
consider how changes in the nutrient conditions, through the parameter $r_{AA}$,
influence the maximum growth rate as determined by \EQ{lambda_limit}.
\FIG{elongation_rate_model}(C) shows how the observed growth rate depends on the
rate of amino acid supply $r_{AA}$ as a function of the cellular ribosome copy
number. A feature immediately apparent is the presence of a maximal growth rate
whose dependence on $R$ (and consequently, the cell size) increases with
increasing $r_{AA}$. Importantly, however, there is an optimum set of $R$,
$N_{pep}$, and $V$ that are strictly dependent on the value of $r_{AA}$.
Increasing the ribosomal concentration beyond the cell's metabolic capacity has
the adverse consequence of depleting the supply of amino acids and a concomitant
decrease in the elongation rate $r_t$ [\FIG{elongation_rate_model}(B)].

Also of note is the growth rate profiles shown for low amino acid supply rates
[purple and blue lines in \FIG{elongation_rate_model}(C)], representing growth
in nutrient-poor media. In these conditions, there no longer exists a peak in
growth, at least in the range of physiologically-relevant  ribosome copy
numbers. Instead, cells limit their pool of actively translating ribosomes  by
decreasing $f_a$ \citep{dai2016}, which would help maintain the pool of
available amino acids $[AA]_\text{eff}$ and increase the achievable elongation
rate. This observation is in agreement with the central premise of the cellular
resource allocation principle proposed by \cite{scott2010,
klumpp2009,klumpp2014} and \cite{hui2015}.
